3D Tisk
3D tisk je proces, při kterém lze z digitálního 3D modelu vytvořit reálný objekt.
Tyto modely je nutné před tiskem nutné předpřipravit.
Jejich příprava spočívá v převedení na soubor instrukcí označovaný jako GCODE.
Tento soubor přesně definuje, jakým způsobem má tiskárna objekt vytvořit.
Objekty vznikají postupným nanášením jednotlivých vrstev na sebe ve vertikálním směru.
Možností jak na sebe materiál lze nanášet je mnoho, jedna z nejpoužívanějších se nazývá Fused deposition modeling (zkráceně FDM).
Ve své bakalářské práci uvažuji pojmem 3D tisk právě tuto metodu.

FDM
Princip FDM spočívá v tavení plastu (případně jiných materiálů) pomocí tiskové hlavy.
Ta následně nanáší taveninu ve vrstvách na sebe.
Informace o tom, jak se má tisková hlava pohybovat v jednotlivých vstvách údává GCODE.
Takto vytvořené plastové objekty mají velkou výhodu v nízkých nákladech na výrobu.
Jsou tedy ideálním prostředkem k výrobě prototypů nebo produkování omezeného množství výrobků.

FDM
GCODE
OctoPrint
