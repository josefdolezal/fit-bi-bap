\section{Doménový model}

Během analýzy koncových bodů OctoPrint jsem sestavoval doménový model.
Tento model pokrývá funkcionalitu koncových bodů, uvažuje také četnosti parametrů a vztahy entit.

Jak je vidět z obrázku \ref{fig:analysis-domain-model}, relace se vyskytují jen mezi některými entitami.
To je dáno tím, že aplikace v jednu chvíli pracuje vždy s právě jednou instancí tiskárny.

\image{fig:analysis-domain-model}{Doménový model aplikace}{assets/analysis-domain-model.pdf}

Objekty týkající se tiskárny se tedy vždy vytváří až při výběru jedné konkrétní.
Po ukončení práce s tiskárnou se opět mažou.

Tento přístup je vhodný, protože informace o tiskárně se pravděpodobně mezi jednotlivými spuštěními aplikace změní.
Uchování starších dat sice působí při používání lépe, protože aplikace nabízí data ihned po otevření, nemusí být ale aktuální, což by vedlo ke zmatení uživatele.

Jediná data, která se uchovávají mezi jednotlivými spuštěními je seznam tiskáren.
