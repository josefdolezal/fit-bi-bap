\section{Synchronizace vláken aplikace} \label{analyza-synchronizace-vlaken}

Mobilní aplikace se svou povahou značně liší od běžných webových nebo desktopových aplikací.
Oproti zmíněným jsou často mnohem interaktivnější, tedy ovládané nejen běžnými vstupy, ale i vlastnostmi zařízení (GPS poloha, orientace zařízení, dostupnost periferií).
Aby tyto vstupy nekazily uživatelský zážitek, jsou v systému implementovány asynchronně.

Jako příklad lze vzít získávání polohy z družic.
To provádí telefon na vlákně v pozadí.
Tím je zaručené, že hlavní vlákno aplikace (které se mimo jiné stará o vykreslování) není blokované a s aplikací lze bez problému intereagovat.
Jakmile jsou dostupné informace o poloze, aplikace zažádá systém o prostředky na hlavním vlákně a získanou polohu prezentuje uživateli.

Tento přístup je v SDK poskytovaným Applem velmi obvyklý a využívá ho mnoho standardních knihoven.
Ne vždy je ale běžné, aby výsledek byl prezentovaný na hlavním vlákně.
Takový princip je využit u síťových požadavků.

Síťový požadavek se vykonává na vlákně v pozadí a na stejném vlákně je zpracována i odpověď ze vzdáleného serveru.
Je-li potřeba odpověď zpracovat na hlavním vlákně, musí vývojář explicitně čas na hlavním vlákně vyžádat pomocí Grand Central Dispatch či vlákna synchronizovat jiným spůsobem.
Od verze systému iOS 9 je navíc nemožné udělat síťový požadavek synchronně (blokovat vlákno, které požadavek vytvořilo až do konce zpracování) \cite{apple-whats-new-in-ios}.

Otázkou tedy zůstává jak správně synchronizovat všechny vlákna, která potřebují zpracovat data na hlavním vlákně.

Při analýze tohoto problému jsem se zaměřil na tři možná řešení.
První dvě jsou implementovány přímo v dodávaném SDK, třetí možnost je knihovna od společnosti GitHub.

\subsection{Grand Central Dispatch}

Grand Central Dispatch (zkráceně GCD) je technologie vyvinutá společností Apple přinášející optimalizovanou podporu pro aplikace fungující na vícejádrových procesorech.
GCD je implementována nad standardními systémovými vlákny, vývojáři ale nabízí mnohem jednoduší rozhranní.

Pro zjednodušení je využit princip front (z angl. queue), které jsou reprenzentovány třídou \textit{DispatchQueue}.
DispatchQueue je implementována jako \textit{thread-safe}, je tedy možné k nim přistupovat ve stejný okamžik z několik vláken najednou \cite{apple-concurrency-programming-guide}.
Do této fronty je možné vkládat jednotky práce, GCD se postará o to aby byly spuštěny na ve správném pořadí.
V závislosti na konfiguraci je pak umožňuje jednotlivé úkoly spouštět synchronně nebo asynchronně.

V základu je dostupná \textit{main} fronta, která je synchronní a umožňuje vykonat práci na hlavním vlákně.
Pro synchronizaci vláken stačí požadované jednotky práce přidávat do správných front  \ref{code:create-dispatch-queue}.

Je-li potřeba vytvořit vlastní frontu (z důvodu uvolnění času na hlavním vlákně), stačí specifikovat název fronty a její prioritu.
Vytvoření nové fronty běžící v pozadí je vidět v ukázce \ref{code:create-dispatch-queue}.

\swiftcode{code:create-dispatch-queue}{GCD: Vytvoření vlastní fronty}{assets/code/create-dispatch-queue.swift}

Přidání do libovolné fronty..

\subsubsection*{Quality of Service}

Pro možnost rozlišení priorit front využívá Apple Quality of Service (zkráceně QoS).
Díky QoS je možné určit, jakou prioritu bude mít danná fronta při rozdělování vláken.
V současné době existují čtyři QoS priority \cite{apple-prioritize-work-with-qos}:

\begin{description}
  \item[User-interactive] Práce, se kterou uživatel přímo interaguje.
  Tato fronta má nejvyšší prioritu, nejčastěji se v ní provádí překreslování uživatelského rozhranní či animace.
  Jednotlivé úkoly z fronty jsou vykonané ihned.
  \item[User-initiated] Práce, kterou zadal uživatel a vyžaduje okamžitý výsledek.
  Nejčastěji se stará o akce, které nastanou po interakci s některým z ovládacích prvků (např. tlačítko).
  \item[Utility] Práce, která vyžaduje více času pro své dokončení.
  Typicky se jedná o síťové požadavky nebo načítání dat.
  \item[Background] Práce s nízkou prioritou, kterou si nevyžádal uživatel.
  Využívá se zejména pro dávkové mazání souborů, synchronizaci nebo indexování databáze.
\end{description}

Při vytváření front je důležité správně volit prioritu.
Budou-li všechny fronty využívat jednu prioritu, může se stát, že systémová vlákna nebudou správně využita.
To by vedlo ke snížení výkonu aplikace a špatné uživatelské zkušenosti.

Přestože GCD je velmi zajímavá abstrakce nad standardními vlákny, jedná se stále o nízkoúrovňové API.
Mezi jednotkami práce nelze vyvářet závislosti či je řetězit.
Tuto možnost ale nábízí \textit{Foundation} framework pomocí \textit{NSOperationQueue}.

\subsection{NSOperationQueue a NSOperation}

NSOperationQueue a NSOperation je abstrakce nad metodou GCD zmíněnou víše.
Toto API je od verze systému iOS 4 implementováno právě pomocí GCD, nabízí ale nové možnosti přístupu k prováděným jednotkám práce.
Hlavním rozdílem oproti GCD je vyloučení pojmu \textit{vlákno}.
Jednotky práce nejsou nadále reprezentovány pomocí \textit{first class funkcí}, ale nově pomocí instancí tříd \textit{NSOperation} o jejichž správu se stará \textit{NSOperationQueue}. \cite{cocoacasts-nsoperation-vs-gcd}

\subsubsection*{NSOperation}

Pro reprezentaci jednotek práce se standardní funkce v GCD nejevily jako dostatečné.
S představením NSOperationQueue, je ale nově možné nad opakovanými jednotkami abstrakci pomocí tříd.
Toho lze docílit pomocí vytváření vlastních tříd dědících od NSOperation.

NSOperation představuje samostatnou jednotku práce (operaci).
Jedná se o abstraktní třídu zajišťující \textit{thread-safe} přístup ke správě stavu, priority a závislostí.

Jako úkol lze chápat například jednotlivé síťové požadavdy, zpracování vstupů a uživatele a jejich perzistence nebo komplexní výpočty.
Úkolem je možné označit ale i libovolnou strukturovanou jednotku práce, u které je potřeba udržovat stav nebo zpracovat její datový výstup.

Oproti CGD má objektový přístup výhodu nejen v přehlednější strukturování, ale i v možnosti vytváření závislostí a správě stavu \cite{apple-operation-queues}.

\begin{description}
  \item[Závislosti] definují, jaké další operace musí být vykonány před tím, než bude daná operace spuštěna.
  Přidat lze libovolný počet dalších operací, tím lze dosáhnout komplexního procesu za pomoci skládání menších bloků.
  Jako příklad lze uvést síťovou komunikaci, kde každý požadavek závisí na ověření dostupnosti internetu (první operace) a na patřičném ověření uživatele (druhá operace).
  Ke každému požadavku následně stačí přiřadit tyto dvě operace jako závislost, tím je zaručené že požadavek se pošle jen když je dostupné připojení k internetu a zároveň je uživatel ověřený.
  \item[Správa stavu] umožňuje operaci pozastavit, spustit znovu nebo zrušit.
  Oproti GCD, kde úkol, který se začal zpracovávat už nebylo možné zastavit, lze například do ovládání operací zapojit uživatele.
\end{description}

\subsubsection*{NSOperationQueue}

Obdobně jako GCD, je i zde potřeba určit kde a kdy se bude práce vykonávat.
Pro tento účel slouží fronta NSOperationQueue.
Tato fronta je implementována jako prioritní.
Tedy v momentě kdy je vložena operace s vyšší prioritou, bude vykonána dříve než stávající operace s nízkou prioritou (pokud to její závislosti dovolí).
Z tohoto odůvodu není zaručeno kdy se jednotlivé operace spustí.

Základní vytvoření operace a vložení do fronty je vidět na ukázce \ref{code:operation-queue-demonstration}.

\swiftcode{code:operation-queue-demonstration}{NSOperationQueue: Vytvoření fronty s operacemi}{assets/code/operation-queue-demonstration.swift}

\subsubsection*{Prioritizace úkolů}

Z důvodu potřeby prioritizace některých úloh je možné jedntlivým operacím nastavit prioritu.
V takovém případě se vždy jako první vykonají operace s vyšší prioritou, následně se přistupuje k těm s nižším.

\subsubsection*{Quality of Service}

Jednou z nejsilněších stránek NSOperationQueue je abstrakce front nad vlákny (tedy celého GCD).
V nastavení fronty lze zvolit stejně jako u GCD výchozí QoS.
Všechny operace pak budou vyžívat právě tuto prioritu.
Oproti GCD má však NSOperationQueue výhodu, že nepracuje pouze s jednou frontou.

To dává vývojáři možnost definovat QoS pro každou operaci zvlášť a to bez nutnosti vytvořát více front.
V závisloti na této vlastnosti je dále možné definovat, kolik operací může být \textit{maximálně} spuštěno v jeden okamžik.

\subsection{Reaktivní programování}\label{vlakna-rac}

\todo{Sekce reaktivní programování}
