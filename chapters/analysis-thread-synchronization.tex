\section{Synchronizace vláken aplikace} \label{analyza-synchronizace-vlaken}

Mobilní aplikace se svou povahou značně liší od běžných webových nebo desktopových aplikací.
Oproti zmíněným jsou často mnohem interaktivnější, tedy ovládané nejen běžnými vstupy, ale i vlastnostmi zařízení (\acrshort{gps} poloha, orientace zařízení, dostupnost periferií).
Aby tyto vstupy nekazily uživatelský zážitek, jsou v systému implementovány asynchronně.

Jako příklad lze vzít získávání polohy z družic.
To provádí telefon na vlákně v pozadí.
Tím je zaručené, že hlavní vlákno aplikace (které se mimo jiné stará o vykreslování) není blokované a s aplikací lze bez problému intereagovat.
Jakmile jsou dostupné informace o poloze, aplikace zažádá systém o prostředky na hlavním vlákně a získanou polohu prezentuje uživateli.

Tento přístup je v \acrfull{sdk} poskytovaným Applem velmi obvyklý a využívá ho mnoho standardních knihoven.
Ne vždy je ale běžné, aby výsledek byl prezentovaný na hlavním vlákně.
Takový princip je využit u síťových požadavků.

Síťový požadavek se vykonává na vlákně v pozadí a na stejném vlákně je zpracována i odpověď ze vzdáleného serveru.
Je-li potřeba odpověď zpracovat na hlavním vlákně, musí vývojář explicitně čas na hlavním vlákně vyžádat pomocí Grand Central Dispatch či vlákna synchronizovat jiným spůsobem.
Od verze systému iOS 9 je navíc nemožné udělat síťový požadavek synchronně (blokovat vlákno, které požadavek vytvořilo až do konce zpracování) \cite{apple-whats-new-in-ios}.

Otázkou tedy zůstává jak správně synchronizovat všechny úkoly, které potřebují zpracovat data na jednom společném vlákně.

Při analýze tohoto problému jsem se zaměřil na tři možná řešení.
První dvě jsou implementovány přímo v dodávaném \acrshort{sdk}, třetí možnost je knihovna od společnosti GitHub.

\subsection{Grand Central Dispatch}

Grand Central Dispatch (zkráceně \acrshort{gcd}) je technologie vyvinutá společností Apple přinášející optimalizovanou podporu pro aplikace fungující na vícejádrových procesorech.
\acrshort{gcd} je implementována nad standardními systémovými vlákny, vývojáři ale nabízí mnohem jednoduší rozhranní.

Pro zjednodušení je využit princip front (z angl. queue), které jsou reprenzentovány třídou \texttt{DispatchQueue}.
\texttt{DispatchQueue} je implementována jako \texttt{thread-safe}, je tedy možné k nim přistupovat ve stejný okamžik z několik vláken najednou \cite{apple-concurrency-programming-guide}.
Do této fronty je možné vkládat jednotky práce, \acrshort{gcd} se postará o to aby byly spuštěny ve správném pořadí.
V závislosti na konfiguraci pak umožňuje jednotlivé úkoly spouštět synchronně nebo asynchronně.

V základu je dostupná hlavní (\texttt{main}) fronta, která je synchronní a umožňuje vykonat práci na hlavním vlákně.
Pro synchronizaci vláken stačí požadované jednotky práce přidávat do správných front.

Je-li potřeba vytvořit vlastní frontu (z důvodu uvolnění času na hlavním vlákně), stačí specifikovat název fronty a její prioritu.
Vytvoření nové fronty běžící v pozadí je vidět v ukázce \ref{code:create-dispatch-queue}.

\swiftcode{code:create-dispatch-queue}{GCD: Vytvoření vlastní fronty}{assets/code/create-dispatch-queue.swift}

\subsubsection*{Quality of Service}

Pro možnost rozlišení priorit front využívá Apple Quality of Service (zkráceně QoS).
Díky QoS je možné určit, jakou prioritu bude mít danná fronta při rozdělování vláken.
V současné době existují čtyři QoS priority \cite{apple-prioritize-work-with-qos}:

\begin{description}
  \item[User-interactive] Práce, se kterou uživatel přímo interaguje.
  Tato fronta má nejvyšší prioritu, nejčastěji se v ní provádí překreslování uživatelského rozhranní či animace.
  Jednotlivé úkoly z fronty jsou vykonané ihned.
  \item[User-initiated] Práce, kterou zadal uživatel a vyžaduje okamžitý výsledek.
  Nejčastěji se stará o akce, které nastanou po interakci s některým z ovládacích prvků (např. tlačítko).
  \item[Utility] Práce, která vyžaduje více času pro své dokončení.
  Typicky se jedná o síťové požadavky nebo načítání dat.
  \item[Background] Práce s nízkou prioritou, kterou si nevyžádal uživatel.
  Využívá se zejména pro dávkové mazání souborů, synchronizaci nebo indexování databáze.
\end{description}

Při vytváření front je důležité správně volit prioritu.
Budou-li všechny fronty využívat jednu prioritu, může se stát, že systémová vlákna nebudou správně využita.
To by vedlo ke snížení výkonu aplikace a špatné uživatelské zkušenosti.

Přestože \acrshort{gcd} je velmi zajímavá abstrakce nad standardními vlákny, jedná se stále o nízkoúrovňové API.
Mezi jednotkami práce nelze vyvářet závislosti či je řetězit.
Tuto možnost ale nábízí \texttt{Foundation} framework pomocí \texttt{OperationQueue}.

\subsection{OperationQueue a Operation}

\texttt{OperationQueue} a \texttt{Operation} je abstrakce nad metodou \acrshort{gcd} zmíněnou výše.
Toto \acrshort{api} je od verze systému iOS 4 implementováno právě pomocí GCD, nabízí ale nové možnosti přístupu k prováděným jednotkám práce.
Hlavním rozdílem oproti \acrshort{gcd} je vyloučení pojmu \texttt{vlákno}.
Jednotky práce nejsou nadále reprezentovány pomocí \texttt{first class} funkcí, ale nově pomocí instancí tříd \texttt{Operation} o jejichž správu se stará \texttt{OperationQueue}. \cite{cocoacasts-nsoperation-vs-gcd}

\subsubsection*{Operation}

Pro reprezentaci jednotek práce se standardní funkce v \acrshort{gcd} nejevily jako dostatečné.
S představením \texttt{OperationQueue}, je ale nově možné nad opakovanými jednotkami práce zakomponovat abstrakci pomocí tříd.
Toho lze docílit pomocí vytváření vlastních tříd dědících od \texttt{Operation}.

\texttt{Operation} představuje samostatnou jednotku práce (operaci).
Jedná se o abstraktní třídu zajišťující \texttt{thread-safe} přístup ke správě stavu, priority a závislostí.

Jako úkol lze chápat například jednotlivé síťové požadavdy, zpracování vstupů a uživatele a jejich perzistence nebo komplexní výpočty.
Úkolem je možné označit ale i libovolnou strukturovanou jednotku práce, u které je potřeba udržovat stav nebo zpracovat její datový výstup.

Oproti \acrshort{gcd} má objektový přístup výhodu nejen v přehlednější strukturování, ale i v možnosti vytváření závislostí a správě stavu \cite{apple-operation-queues}.

\begin{description}
  \item[Závislosti] definují, jaké další operace musí být vykonány před tím, než bude daná operace spuštěna.
  Přidat lze libovolný počet dalších operací, tím lze dosáhnout komplexního procesu za pomoci skládání menších bloků.
  Jako příklad lze uvést síťovou komunikaci, kde každý požadavek závisí na ověření dostupnosti internetu (první operace) a na patřičném ověření uživatele (druhá operace).
  Ke každému požadavku následně stačí přiřadit tyto dvě operace jako závislost, tím je zaručené že požadavek se pošle jen když je dostupné připojení k internetu a zároveň je uživatel ověřený.
  \item[Správa stavu] umožňuje operaci pozastavit, spustit znovu nebo zrušit.
  Oproti GCD, kde úkol, který se začal zpracovávat už nebylo možné zastavit, lze například do ovládání operací zapojit uživatele.
\end{description}

\subsubsection*{OperationQueue}

Obdobně jako GCD, je i zde potřeba určit kde a kdy se bude práce vykonávat.
Pro tento účel slouží fronta \texttt{OperationQueue}.
Tato fronta je implementována jako prioritní.
Tedy v momentě kdy je vložena operace s vyšší prioritou, bude vykonána dříve než stávající operace s nízkou prioritou (pokud to její závislosti dovolí).
Z tohoto odůvodu není zaručeno kdy se jednotlivé operace spustí.

Základní vytvoření operace a vložení do fronty je vidět na ukázce \ref{code:operation-queue-demonstration}.

\swiftcode{code:operation-queue-demonstration}{OperationQueue: Vytvoření fronty s operacemi}{assets/code/operation-queue-demonstration.swift}

\subsubsection*{Prioritizace úkolů}

Z důvodu potřeby prioritizace některých úloh je možné jednotlivým operacím nastavit prioritu.
V takovém případě se vždy jako první vykonají operace s vyšší prioritou, následně se přistupuje k těm s nižší.

\subsubsection*{Quality of Service}

Jednou z nejsilněších stránek \texttt{OperationQueue} je abstrakce front nad vlákny (tedy celého GCD).
V nastavení fronty lze zvolit stejně jako u \acrshort{gcd} výchozí QoS.
Všechny operace pak budou vyžívat právě tuto prioritu.
Oproti \acrshort{gcd} má však \texttt{OperationQueue} výhodu, že nepracuje pouze s jednou frontou.

To dává vývojáři možnost definovat QoS pro každou operaci zvlášť a to bez nutnosti vytvářet více front.
V závislosti na této vlastnosti je dále možné definovat, kolik operací může být \textit{maximálně} spuštěno v jeden okamžik.

\subsection{Reaktivní programování}\label{vlakna-rac}

Reaktivní programování získává poslední dobou velký zájem vývojářů \cite{oneagency-rx}.
Jedná se o programování založené na reakci na uživatelské vstupy.
Nejčastěji je tento princip spojován s programováním uživatelského rozhraní, protože je implementováno asynchronně a neblokuje tedy hlavní vlákno aplikace.

Myšlenka reaktivního programování je nepřistupovat k datům jako stálým hodnotám, ale jako k proudu (streamu) hodnot v konkrétní čas.
Implementací existuje více, např. Promise \cite{slaks-promise}, Callback \cite{yld-callback} či funkcionálně reaktivní programování.
V analýze jsem zabýval pouze funkcionálně reaktivním programováním, konkrétně imlementací v knihovně \texttt{ReactiveCocoa}.

\subsubsection*{Funkcionálně reaktivní programování -- \acrshort{frp}}

Funkcionálně reaktivní programování (zkráceně \acrshort{frp}) označuje stream hodnot jako signál.
Z teorie funkcionálního programování se k signálům přistupuje jako ke konstantám.
Jednou vytvořený signál nelze měnit ani ho ovlivnit vedlejšími účinky.
Je-li potřeba signál změnit, je nutné vytvořit nový signál a ten používat místo původního.

Signál může reprezentovat libovolnou událost v aplikaci: stisknutí tlačítka, síťový požadavek, změny hardware.
Hodnota je v signálu dostupná pouze ve chvíli, kdy je signálem poslána a doručena objektům, které se zaregistrovali o její příjem.
Těmto objektům se v terminologi \texttt{ReactiveCocoa} říká \texttt{Observer}.
Signály tedy nepodporují přímý přístup k hodnotám.

Přestože se absence přímého přístupu může zdát jako velké negativum, v praxi to není žádný problém.
Místo standardního přístupu k hodnotám se využívá kombinování, řetězení a přetváření signálů.
V aplikaci pak není nutné využívat pull-based přístup pro získání dat.
Jakmile jsou totiž nová data dostupná, zpropagují se na patřičná místa podle předem namodelovaného procesu.
Vytvoření signálu je vidět v ukázce \ref{code:rac-signal}.

\swiftcode{code:rac-signal}{Vytvoření signálu v ReactiveCocoa}{assets/code/rac-signal.swift}

\subsubsection*{Hot a Cold signály}

Z ukázky \ref{code:rac-signal} je vidět, že hodoty prochází signálem okamžitě.
Zde nastává problém ve chvíli, kdy se \texttt{Observer} o přijímání hodnot zaregistruje až po té, co nějaké hodnoty už signálem proběhly.
V \texttt{ReactiveCocoa} jsou proto dva typy signálů.

\texttt{Hot} signál reprezentovaný třídou \texttt{Signal} odesílá hodnoty okamžitě a neudržuje jejich historii.
Pokud se \texttt{Observer} zaregistruje pozdě, hodnotu nikdy neobdrží.
U síťového požadavku by se mohlo stát, že přijde odpověď od serveru, ale nebude zpracována.
Tento typ se tedy využívá v částech programu, kde absence některých hodnot nezpůsobí nekonzistenci.

Druhým typem signálu je \texttt{Cold} signál reprezentovaný třídou \texttt{SignalProducer}.
Tato třída odesílá hodnoty až ve chvíli, kdy se o ně nějaký \texttt{Observer} přihlásí.
Tomuto \texttt{Observeru} navíc budou odeslány postupně všechny hodnoty, které byly do tohoto signálu vloženy dříve jak je vidět v ukázce \ref{code:rac-signalproducer}.
To je ideální přístup pro manipulaci s daty či síťové požadavky.

\swiftcode{code:rac-signalproducer}{Vytvoření Cold signálu}{assets/code/rac-signalproducer.swift}

\subsubsection*{Operátory nad signály}

Protože signály jsou konstantní a jejich hodnotu nelze editovat, nabízí \texttt{ReactiveCocoa} operátory nad signály.
Pomocí těchto operátorů je možné vytvořit nový signál stejného typu, jehož hodnota vychází z původní hodnoty libovolnou úpravou.

Tyto operátory přejímají názvy z teorie funkcionálního programování.
Dostupné jsou tedy operátory \texttt{map}, \texttt{filter}, \texttt{reduce} a další.

Pomocí rozšíření tříd dostupných v jazyku Swift je také možné dělat vlastní operátory.

\subsubsection*{Synchronizace vláken}

Pomocí signálů lze jednoduše zařídit synchronizaci vláken.
K tomu jsou dostupné knihovní funkce \texttt{flatMap}, \texttt{combineLatest}, \texttt{merge} a další.
Signály mohou vzniknout na různých vláknech, knihovna se ale stará o to, aby k signálům bylo přistupováno vždy právě z jednoho vlákna.
Každý \texttt{Observer} si také může zvolit, na jakém vlákně mu budou hodnoty předávány.

Více informací ke knihovně \texttt{ReactiveCocoa} a jejímu využití v reaktivním programování je dostupné na její GitHub stránce \cite{github-reactivecocoa}.
