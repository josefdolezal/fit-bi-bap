\section{Správa závislostí}\label{analyza-sprava-zavislosti}

Pro zakomponování knihoven třetích stran jsem se rozhodl využít správce závislostí.
Pomocí správce závislostí je možné jednoduše udržovat knihovny aktuální či s projektem svázat konkrétní verzi knihovny.
Při analyzování nástrojů na správu závislostí jsem se zabýval dvěma nejpoužívanějšími \cite{shashikantjagtap-swift-dependency-management}.
Prvním nástrojem jsou \textit{CocoaPods} \cite{cocoapods-about}, open source nástroj řízený komunitou.
Druhým nástrojem je \textit{Carthage} \cite{github-carthage}, vytvořený společností GitHub a nyní vedený jako open source.

\subsection{CocoaPods}

CocoaPods je nástroj s rozhraním pro příkazovou řádku, který umožňuje spravovat závislosti pro Xcode projekty.
Základem jsou takzvané \textit{Pods}, tedy balíčky závislostí.
Tyto balíčky může uživatel integrovat do svého projektu pomocí souboru \textit{Podfile}, do kterého vypíše seznam seznam závislostí.
CocoaPods následně závislosti stáhne a vytvoří pracovní \textit{workspace}, do kterého závislosti nalinkuje.
Pro následné programování pak uživatel využívá nový workspace místo původního souboru projektu.

Nevýhoda tohoto přístupu je, že jsou staženy přímo zdrojové kódy závislostí a ty jsou připojeny k projektu.
Z tohoto důvodu se při kompilaci může stát, že se kompilují soubory nejen aplikace, ale i všech závislostí.
Tím se zvyšuje čas potřebný ke zkompilování aplikace.
To má za následek snížení rychlosti vývoje.

\subsection{Carthage}

Druhým analyzovaným nástrojem je Carthage.
Obdobně jako Cocoapods nabízí rozhraní příkazové řádky.
Hlavním je rozdílem je způsob integrace.
Carthage žádným způsobem není integrován do projektu jako takového.
Jedinou funkcí tohoto správce závislostí je stáhnout a zkompilovat zdrojové kódy požadovaných závislostí.
Uživatel následně musí sám knihovny do projektu naimportovat.

Pro urychlení kompilace mohou správci knihoven předkompilovat knihovny do binární formy, ta se následně pouze stáhne.
Předkompilované knihovny ale musí být napsané ve stejné verzi Swiftu jako používá aplikace sama, v opačném případě nepůjde knihovny slinkovat ani zkompilovat aplikaci.
Takový případ není v lokálním vývoji velkou komplikací, může ale narušit funkcionalitu integračního procesu na vzdáleném serveru.
I to je ale situace, která se vyskytne pouze v době vydání nové verze nástroje Xcode.
To bývá zpravidla jen několikrát ročně.
