\section{Architektura aplikace}\label{analyza-architektura}

Jako doporučenou architekturu aplikací pro platformu iOS uvádí Apple Model-View-Controller (zkráceně \acrshort{mvc}) \cite{gc-apple-recommends}.
Přestože je \acrshort{mvc} pro vývoj aplikací nejpopulárnější, rozhodl jsem před začátkem implementace prozkoumat i jiné existující architektury.
Z alternativních architektur jsem nakonec zvolil Model-View-ViewModel, kterou porovnám s doporučeným \acrshort{mvc}.
V závislosti na výsledku porovnání zvolím ideální architekturu pro svou aplikaci.

\subsection{\acrshort{mvc}: Model-View-Controller}\label{analyza-mvc}
Tato architektura rozděluje aplikaci do tří vrstev: \texttt{Model}, \texttt{View} a \texttt{Controller}.
Názvy vrstev se běžně do českého jazyka nepřekládají.
Jejich významu se věnuji níže.

\subsubsection*{Popis architektury}

\begin{figure}\centering
	\includegraphics[width=0.75\textwidth]{assets/analysis-mvc-architecture.pdf}
	\caption{Architektura MVC}\label{fig:architektura-mvc}
\end{figure}

\begin{description}
  \item[Model] Reprezentuje perzistentní objekty, které aplikace využívá pro vnitřní logiku a prezentaci dat uživateli.
  Každý modelový objekt může být v relaci s libovolným počtem jiných modelových objektů.
  Tato vrstva je často reprezentována databází, příkladem mohou být databáze \texttt{Core Data}, \texttt{Realm} nebo \texttt{SQLite}.
  Ukázka \ref{code:mvc-model} prezentuje možnou podobu modelového objektu.
  \swiftcode{code:mvc-model}{Modelový objekt v architektuře \acrshort{mvc}}{assets/code/mvc-model.swift}

  \item[View] Datový objekt viditelný uživatelem. \texttt{View} obsahuje logiku pro vykreslení a interakci s uživatelem.
  Přestože se \texttt{View} standardně používá pro zobrazení \texttt{Model} objektů nebo jejich úpravu, jsou od sebe tyto vrstvy striktně odstíněny.
  Na platformě iOS tuto vrstvu reprezentuje framework \texttt{UIKit} vytvořený Applem.
  Ukázkový kód \ref{code:mvc-view} zobrazuje možnou implementaci \texttt{View}.

  \item[Controller] Aplikační vrstva, která na základě vstupů z \texttt{View} aktualizuje a mění \texttt{Model} nebo překresluje \texttt{View} v případě, že zobrazovaná data už nejsou aktuální.
  Jedním z úkolů \texttt{Controlleru} je striktně zamezit přímé interakci mezi \texttt{View} a \texttt{Modelem}.
  Toto oddělení je zavedeno proto, aby \texttt{View} nemuselo znát konkrétní strukturu \texttt{Modelu} a aby \texttt{Model} nemusel obsahovat logiku formátování dat (cena, čas, ...) pro vykreslení.
  Dále se stará o navigaci mezi obrazovkami, síťování a interakci s uživatelem.
  Při rozdělení do obrazovek platí pravidlo, že jeden \texttt{Controller} obsluhuje jedno nebo více \texttt{View}.
  Ke korektnímu vykreslení \texttt{View} využívá libovoné množství modelových objektů.
  O jednu obrazovku se typicky stará právě jeden \texttt{Controller}, je ale možné jich použít více.
  Zjednodušenou implementaci lze vidět v ukázce \ref{code:mvc-view-controller}.
\end{description}

Z tohoto shrnutí vyplývá, že \texttt{Controller} je velmi blízce spjat s \texttt{View}. Toto propojení reprezentuje obrázek \ref{fig:massive-mvc}.

\subsubsection*{\texttt{Modelový} příklad použití}

Pro možné porovnání architektury jsem připravil scénář stažení libovolných dat na základě požadavku uživatele.
V \acrshort{mvc} by se architektura chovala takto:

\begin{itemize}
  \item Uživatel v aplikaci klikne na tlačítko \uv{Stáhnout data}.
  \item Tuto interakci odchytí \texttt{View} a upozorní \texttt{Controller}.
  \item \texttt{Controller} na základě upozornění stáhne data a předá je \texttt{Modelu} k uložení.
  \item \texttt{Model} ukládá data a notifikuje \texttt{Controller} o změně.
  \item \texttt{Controller} aktualizuje \texttt{View}.
  \item Nastane-li během stahování chyba, \texttt{Controller} vytváří nové \texttt{View} a chybu prezentuje uživateli.
\end{itemize}

\swiftcode{code:mvc-view}{View v architektuře \acrshort{mvc}}{assets/code/mvc-view.swift}

\subsubsection*{Shrnutí}

Shrneme-li vlastnosti vrstev, jejich klíčové role jsou:

\begin{itemize}
  \item \texttt{Model} udává, jakým způsobem jsou data uložena,
  \item \texttt{View} se stará o správné vykreslení předformátovaných dat,
  \item \texttt{Controller} se stará o ostatní logiku.
\end{itemize}

\swiftcode{code:mvc-view-controller}{ViewController v architektuře \acrshort{mvc}}{assets/code/mvc-view-controller.swift}

Pro zmíněné notifikace nabízí Apple řešení pomocí Delegate pattern návrhový vzor.
Tento návrhový vzor udává, že \texttt{Controller} musí naimplementovat specifické rozhraní, čímž se stane delegátem.
Jako delegát se pak může zaregistrovat na notifikace obejktů, jejichž rozhraní implementoval.

\acrshort{mvc} je v době psaní této práce nejpoužívanější architekturou a to především díky své jednoduchosti.
Při tvorbě větších aplikací ale nemusí být vhodné.
\texttt{Controller} se při nestandradním grafickém návrhu může stát velmi složitým, což výrazně snižuje jeho čitelnost a testovatelnost.
Z tohoto důvodu se \acrshort{mvc} často přezdívá \uv{Massive \texttt{View} Controller}.
Kvůli přímému napojení \texttt{Controlleru} na \texttt{View} se při testování jeho chování (behavior testing) musí využít simulátoru mobilního operačního systému.
Pro simulátor je navíc nutné naskriptovat uživatelův průchod aplikací, aby bylo možné testy automatizovat.

To zvyšuje časovou náročnost testování, dokonce v některých případech znemožňuje testování úplně (\texttt{Controlleru} nezle podvrhnout mock objekty).
Tento problém se snaží řešit architektura \acrfull{mvvm} od společnosti Microsoft.

\begin{figure}\centering
	\includegraphics[width=0.75\textwidth]{assets/analysis-massive-view-controller.pdf}
	\caption[Role Controlleru v MVC]{Controller spjatý s View}\label{fig:massive-mvc}
\end{figure}

\subsection{MVVM: Model-View-ViewModel}\label{analyza-mvvm}

Z důvodu nárustu nároků na mobilní aplikace se v posledních letech rozmáhá architektura \acrshort{mvvm}.
Tato architektura vychází ze zmíněného \acrshort{mvc} a jejím základním úkolem je zjednodušit \texttt{Controller}.

\subsubsection*{Popis architektury} \label{architektura-mvvm-popis}

Za účelem zjednodušení \texttt{Controlleru} se ke stávajícím třem vrstvám přidává \texttt{ViewModel}, který se stará o přípravu dat z Modelu pro zobrazení a také o perzistenci změn.

\begin{description}
  \item[ViewModel] je objekt vlastněný \texttt{Controllerem} za pomoci kompozice.
  Pro \texttt{Controller} připravuje naformátované výstupy a poskytuje mu rozhraní pro vstupy.
  Výstupem se rozumí veškerá data, která jsou potřebná pro sestavení \texttt{View}.
  To může být např. datum ve specifickém formátu, cena včetně měny nebo informace o tom, kolik řádků bude obsahovat tabulka na obrazovce.
  Oproti \acrshort{mvc} tedy perzistentní data nejsou viditelná \texttt{Controlleru}, ale pouze \texttt{ViewModelu}.
  Ten je nejdříve připraví pro zobrazení.
  Vstupem může být libovolná interakce uživatele:
  změna textu v textovém poli, stisknutí tlačítka, ale i fyzický pohyb telefonem (otočení obrazovky).
  Na základě vstupů spouští \texttt{ViewModel} svou vnitřní logiku a generuje výstupy.
  Jednoduchou implementaci \texttt{ViewModelu} bez použití vstupů lze vidět v ukázce \ref{code:mvvm-view-model}.
\end{description}

\swiftcode{code:mvvm-view-model}{ViewModel v architektuře \acrshort{mvvm}}{assets/code/mvvm-view-model.swift}

Zodpovědnost \texttt{Controlleru} se zavedením \texttt{ViewModelu} dramaticky snižuje.
V ideálním případě je \texttt{Controller} zodpovědný pouze za správné sestavení \texttt{View} a napojení zfromátovaných výstupů na něj.
Dále pak za odchycení uživatelských interakcí a jejich propagaci do \texttt{ViewModelu}.
Toto chování zachycuje obrázek \ref{architektura-mvvm}.

\begin{figure}\centering
	\includegraphics[width=0.9\textwidth]{assets/analysis-mvvm-architecture.pdf}
	\caption[Architektura \acrshort{mvvm}]{Architektura \acrshort{mvvm}}\label{architektura-mvvm}
\end{figure}

\subsubsection*{Modelový příklad použití} \label{architektura-mvvm-priklad}

Pro porovnání architektury s \acrshort{mvc} lze opět využít scénář pro stažení dat.
Pro tento scénář by se architektura \acrshort{mvvm} chovala následovně:
\begin{itemize}
  \item \texttt{Controller} napojuje výstupy \texttt{ViewModelu} na \texttt{View} a vytváří pravidla pro převod uživatelské interakce na vstupy \texttt{ViewModelu}.
  \item Uživatel v aplikaci klikne na tlačítko \uv{Stáhnout data}.
  \item \texttt{View} upozorňuje \texttt{Controller} na interakci uživatele, ten automaticky vytváří vstup pro \texttt{ViewModel}.
  \item \texttt{ViewModel} na základě vstupu stahuje data a předává je \texttt{Modelu}.
  \item \texttt{Model} po uložení notifikuje \texttt{ViewModel}, ten vytváří výstup pro \texttt{Controller}, který nechává překreslit \texttt{View}.
  \item V případě chyby vytvoří \texttt{ViewModel} chybový výstup, ten se pomocí \texttt{Controlleru} propaguje do \texttt{View}.
\end{itemize}

\subsubsection*{Shrnutí} \label{architektura-mvvm-shrnuti}

Vrstvy mají následující klíčové vlastnosti:
\begin{itemize}
  \item \texttt{Model} definuje jakým způsobem jsou data uložena a při změně notifikuje \texttt{ViewModel},
  \item \texttt{View} vykresluje na obrazovku naformátované výstupy a upozorňuje \texttt{Controller} při interakci uživatele,
  \item \texttt{Controller} sestavuje hierarchii \texttt{View}, napojuje zformátované výstupy \texttt{ViewModelu} na \texttt{View} a z uživatelské interakce vytváří vstupy pro \texttt{ViewModel},
  \item \texttt{ViewModel} načítá data \texttt{Modelu}. Na základě vstupů z \texttt{Controlleru} nebo změny \texttt{Modelu} generuje výstupy pro \texttt{Controller}.
\end{itemize}

Oproti \acrshort{mvc} je na tomto příkladu vidět snížení zodpovědnosti \texttt{Controlleru}.
Tato zodpovědnost se přesunula do \texttt{ViewModelu} (viz ukázka \ref{code:mvvm-view-controller}).
Na první pohled nemusí být tato změna opodstatněná, protože logika aplikace nezmizela, jen se přesunula.
Právě to ale umožnilo (nebo minimálně zjednodušilo) způsob, jakým lze logiku testovat.
\texttt{ViewModel} generuje výstupy na základě vstupů, v testech tedy lze uživatelskou interakci podvrhnout a testovat pouze výstupy (není potřeba vytvářet \texttt{View} ani \texttt{Controller}).
Dodatečně lze otestovat i uživatelské rozhraní.
Protože logika aplikace je otestována pomocí testů \texttt{ViewModelu}, uživatelské rozhraní už stačí otestovat např. shodou \texttt{View} s referenčním obrázkem.

\swiftcode{code:mvvm-view-controller}{Controller v architektuře \acrshort{mvvm}}{assets/code/mvvm-view-controller.swift}

Při pohledu na notifikace je vidět, že přibyl typ, který nebyl v \acrshort{mvc} potřeba.
Jedná se o notifikace směrem z \texttt{ViewModelu} ke \texttt{Controlleru} (\texttt{ViewModel} nemá referenci na \texttt{Controller}, nemůže ho notifikovat přímo).
Některé výstupy \texttt{ViewModelu} je tedy potřeba sledovat v čase a na jejich změny reagovat.
Toto lze vyřešit pomocí \acrfull{kvo}, které nabízí Apple mezi standardními knihovnami.
\acrshort{kvo} umožňuje objektu zaregistrovat se na notifikace o změně stavu nějakého libovolného jiného objektu.
V případě \texttt{Controlleru} by se registroval na změny stavu výstupů \texttt{ViewModelu}.
Kdykoliv by se výstup změnil, \texttt{Controller} by dostal notifikaci.
Tento přístup ale není běžný pro použití s jazykem Swift.
Tento postup navíc neřeší synchronizaci vláken, z tohoto důvodu by mohlo docházet k nekonzistenci dat či neočekávanému chování.
Místo \acrshort{kvo} se nyní standardně používají reaktivní rozšíření, které popisuji v následujících kapitolách.

Přestože mnou implementovaná aplikace není v ohledu na uživatelské scénáře nijak složitá, obsahuje mnoho obrazovek.
Obrazovky jsou vysoce interaktivní a více se k jejich implementaci hodí reaktivní přístup.
