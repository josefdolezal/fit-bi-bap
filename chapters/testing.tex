\chapter{Testování}

V kapitole o testování se podrobněji zabývám způsoby a postupy naznačenými v kapitolách Architektura MVVM \ref{chapter-mvvm} a Reaktivní programování \ref{chapter-reactive-programming}.

Testováním software se rozumí postupy a procesy, pomocí kterých lze měřit, zda testovaný software (či jeho části) splňuje požadované nároky či nikoliv.
Opakovaným aplikováním těchto postupů lze v softwaru nalézt chyby, nedostatky nebo chybějící vlastnosti oproti dodané specifikaci.
Výsledky testování následně vypovídají o kvalitě softwaru a o míře splnění specifikace. \cite{software-testing-definition}

V této práci jsem využil tří typů testů.
Prvním typem jsou uživatelské testy, zkoumající chování uživatelů při používání aplikace.
Druhým typem jsou testy uživatelského rozhraní, které zjišťují, zda vlivem změn ve vzhledu aplikace nebyla omezena některá z vyžadovaných funkcionalit.
Posledním technika testování, kterou jsem využil jsou testy chování aplikace.
Ty ověřují zda aplikace na sadu vstupů produkuje odpovídající výstupy.
V následujícím shrnutí se těmto technikám věnuji podrobněji.

\section{Uživatelské testy}

Slouží ke zjišťování problémů s návrhem vzhledu aplikace.
Na výsledcích těchto testů lze sledovat, jaké části aplikace jsou uživatelům nesrozumitelné či zda se v aplikaci orientují.
Obsahem testů jsou předpřipravené scénáře pokrývající vybrané funkce aplikace.\cite{dobry-web-ux-testing}

Tyto testy se provádějí na vzorku vybraných uživatelů z cílové skupiny.
Uživatelé během testů prochází aplikaci a mají za úkol naplnit zadanané scénáře.
Během testů se podrobně sleduje, jakým způsobem uživatelé reagují a jak úspěšní jsou při plnění úkolů.
Vzhledem k vysoké časové (mnohdy také finanční) náročnosti se tyto testy provádějí obvykle pouze jednou a to v začátcích projektu. \cite{h1-ux-testing}

Ve své práci jsem prováděl testování na malé skupině uživatelů 3D tiskáren.
Na základě pozorování jsem následně změnil...\textbf{TODO}

\section{Testy uživatelského rozhraní}

Dávají možnost intereagovat s komponentami uživatelského rozhraní za účelem jejich validace.
Na rozdíl od testů chování přistupují tyto testy k aplikaci jako k celku a zacházejí s ní obdobně jako by s ní zacházel uživatel. Tyto testy tedy nemají přístup k vnitřní implementaci aplikace.
Jelikož nevyžadují během chodu zásah člověk (test *nahrazuje* jeho přítomnost), mohou být pouštěny automaticky.
Standardně se tedy pouští při implementaci každé nové funkce, mnohdy až několikrát denně. \cite{apple-ui-testing}

Protože tyto testy z jsou v mém případě pouze nadstavbou nad *testy chování* vysvětlené níže, rozhodl jsem se je implementovat pomocí referenčních obrázků.
Testy tedy pro každý podstatný krok scénáře obsahují referenční obrázek, jak by obrazovka měla v dannou chvíli vypadat.
Pokud se vzhled s referenčním obrázkem shoduje, test projde.
Nevýhodou tohoto přístupu je nutnost přegenerování referenčních obrázků v momentě,
kdy se vzhled obrazovky (úmyslně) změní byť o jediný obrazový bod.
Podstatnou výhodou tohoto přístupu je ale nezávislost na implementaci.
Pokud se implementace změní, s velkou pravděpodobností to výsledky testů neovlivní.

\section{Testy chování}

estované objekty mají definované své rozhraní a závisloti.
Tím jsou deklarovány i závazky objektu.
Závazky určují, jakým způsobem by měl objekt působit na zbylé části aplikace, jaké schopnosti a funkce má - definují chování objektu.
Testovaným objektem může být libovolný modul aplikace.
Složitost a rozsah testů se odvíjí od počtu závazků objektu.
Testování probíhá tak, že vlastnosti objektu jsou pomocí rozhraní měněny a sleduje se, jestli se objekt na základě změn chová podle očekávání. \cite{objcio-bdd}


V reálném světě si lze jako objekt představit auto.
Jako změnu vlastnosti můžeme lze použít vyprázdnění nádrže.
Očekávaným chováním auta následně je, že přestane být pojízdné a rozsvítí kontrolku řidiči.
Auto projde testem chování, právě když při prázdné nádrži je nepojízdné a svítí kontrolka.


Tento princip testování využívám pro testování View Modelu a Modelu (viz. kapitola MVVM: Model-View-ViewModel).
Díky úspěšnému otestování více než 70\% [TODO] logiky aplikace, lze usuzovat, že při používání aplikace se chyb vyskytne jen nepatrné množství.
View vrstvu následně není samostatně nutné testovat, protože z testů View Modelu je jisté, že data jsou správně připravena k zobrazení.
Testy pomocí porovnání skutečného vzhledu s očekávaným (zmíněno v testech uživatelského rozhraní) jsou tedy dostačující.


Obdobně jako testy rozhraní i tyto testy se pouštějí při vývoji až několikrát denně.
Slouží také jako dobrý ukazatel kvality aplikace a dokáží určit její rozsah.


S pojmem *testování chování* je velmi blízce spjat **vývoj řízený testováním chování** (z anglického Behavior Driven Development, zkráceně BDD).
V tomto přístupu k vývoji se před konkrétní implementací nejdříve nadefinuje, jakým způsobem se mají testované komponenty chovat.
Následně se sestavují testy, které vyžadují úplné implementování vyžadovaného chování.
Tím má objekt definované, jaké rozhraní musí implementovat a jaké závislosti bude vyžadovat.
Testy jsou popisovány tkzv. DSL, který pomocí kombinace klíčových slov a textového popisu chování. \textbf{SHOW PREVIEW DSL}
Na základě definice objektu se následně přechází ke konkrétní implementaci. \cite{objcio-bdd}
