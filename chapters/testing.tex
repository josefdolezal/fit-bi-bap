\chapter{Testování}\label{testovani}

\todo{Aktualizovat kapitolu o testech}

\todo{Využití dockeru}

V kapitole o testování se podrobněji zabývám způsoby a postupy naznačenými v kapitolách \ref{analyza-mvvm} a \ref{vlakna-rac} o architektuře MVVM a reaktivní programování.

Testováním software se rozumí postupy a procesy, pomocí kterých lze měřit, zda testovaný software (či jeho části) splňuje požadované nároky či nikoliv.
Opakovaným aplikováním těchto postupů lze v softwaru nalézt chyby, nedostatky nebo chybějící vlastnosti oproti dodané specifikaci.
Výsledky testování následně vypovídají o kvalitě softwaru a o míře splnění specifikace. \cite{software-testing-definition}

V této práci jsem využil dva typů testů.
Jako první techniku testování jsem zvolil testování chování (z anglického \textit{behavior tests}).
Ty ověřují zda aplikace na sadu vstupů vrací odpovídající výstupy.
Další technikou jsou testy uživatelského rozhraní.
Ty zjišťují, zda během vývoje nových funkcionalit nedošlo k poškození stávajícího uživatelského rozhraní.
V následujícím shrnutí se těmto technikám věnuji podrobněji.

\section{Testy chování}\label{testovani-bdd}

Chování modolu aplikace lze otestovat tak, že se zkoumá plnění jeho závazků.
Testované objekty mají definované rozhraní a závisloti, tím mají také deklarovány i závazky.
Závazky určují, jakým způsobem by měl objekt působit na zbylé části aplikace a jaké schopnosti a funkce má.
Ze schopností a funkcí lze odvodit jakým způsobem se má objekt chovat.

\subsection{Testování komponent}

Pomocí těchto testů lze ověřovat funkčnost libovolné komponenty aplikace.
Složitost a rozsah těchto testů se následně odvíjí od počtu závazků komponenty.
Testování probíhá tak, že vlastnosti objektu jsou pomocí rozhraní měněny.
Přitom se sleduje, jestli se objekt na základě změn chová podle očekávání. \cite{objcio-bdd}

\subsection{DSL}

Testy jsou popisovány takzvaným \acrshort{dsl} jazykem.
To je jazyk který pomocí kombinace klíčových slov a textového popisu definuje jak se komponenta má v určitou chvíli chovat.
Více o \acrshort{dsl} lze zjistit na \cite{petrikainulainen-dsl}.

V běžném testovacím rozhraní prostředí Xcode tento jazyk nenabízí.
Rozhodl jsem pro to využí dvou knihoven.
Pomocí Knihovny \textit{Quick} jsem mohl \acrshort{dsl} v projektu využít.
Očekávané hodnoty jsem ověřoval pomocí \textit{Nimble}.

\subsection{Testování ViewModelu}

Tento princip využívám ve své implementaci pro testování Modelu a ViewModelu.
Díky dostatečnému rozsahu testů logiky aplikace lze usuzovat, že v produkčním nasazení se vyskytne jen nepatrné množství chyb.
View vrstvu následně není samostatně nutné testovat, protože z testů ViewModelu je téměř jisté, že data jsou správně připravena k zobrazení.

U ViewModelu jsem podle zvoleného scénáře navrhl \acrshort{dsl}.
Tímto jazykem jsem definoval jak očekávám, že s objektem bude pracovat.
Pomocí Nimble knihovny jsem dále ověřoval, jestli na vytvořené vstupy vytváří ViewModel očekávané výstupy jak je vidět v ukázce \ref{code:bdd-dsl}.

Během implementace jsem vytvořil bez mála sto testů, díky kterým jsem odchytil velké množství chyb.
Tyto chyby jsem často do aplikace zanesl v momentě, kdy jsem k již hotové obrazovce implementoval novou funkcionalitu.

\swiftcode{code:bdd-dsl}{Použití \acrshort{dsl} pro definici testu}{assets/code/bdd-dsl.swift}

\subsection{Vývoj řízený testováním chování}

S pojmem testování chování je velmi blízce spjat \textit{vývoj řízený testováním chování} (z anglického Behavior Driven Development, zkráceně \acrshort{bdd}).
V tomto přístupu k vývoji se před konkrétní implementací nejdříve nadefinuje, jakým způsobem se mají testované komponenty chovat.
Následně se sestavují testy, které vyžadují implementování vyžadovaného chování.
Tím má objekt definované, jaké rozhraní musí implementovat a jaké závislosti bude vyžadovat.

Průběh vývoje probíhá cyklicky.
Nejdříve se nadefinují testy některé funkcionality, následně se implementuje jen tolik kódu, kolik je nutné aby testy byly úspěšné.
Tento postup se opakuje dokud nejsou implementovány veškeré funkce vyžadované po objektu.
Více informací o tomto přístupu je dostupných na \cite{objcio-bdd}.

Tento přístup jsem využil na část ViewModelů.
Vzhledem k rozsahu implementace jsem ale většinu testů napsal až po kompletní implementaci ViewModelu.


\section{Testy uživatelského rozhraní}\label{testovani-ui}

\todo{Přepsat, přidat informace o snapshot testech}

Testování uživatelského rozhraní si klade za cíl ověřit správné sestavení komponent grafického rozhraní.
Pomocí interakce s komponentami se také zkoumá, jakým způsobem komponenty reagují.
Na rozdíl od testů chování přistupují tyto testy k aplikaci jako k celku a zacházejí s ní obdobně jako by s ní zacházel uživatel. Tyto testy tedy nemají přístup k vnitřní implementaci aplikace.
Jelikož nevyžadují během chodu zásah člověka (test \textit{nahrazuje} jeho přítomnost), mohou být pouštěny automaticky.
Standardně se tedy pouští při implementaci každé nové funkce, mnohdy až několikrát denně. \cite{apple-ui-testing}

Protože tyto testy z jsou v mém případě pouze nadstavbou nad \textit{testy chování} vysvětlené níže, rozhodl jsem se je implementovat pomocí referenčních obrázků.
Testy tedy pro každý podstatný krok scénáře obsahují referenční obrázek, jak by obrazovka měla v danou chvíli vypadat.
Pokud se vzhled s referenčním obrázkem shoduje, test projde.
Nevýhodou tohoto přístupu je nutnost přegenerování referenčních obrázků v momentě, kdy se vzhled obrazovky (úmyslně) změní byť o jediný obrazový bod.
Podstatnou výhodou tohoto přístupu je ale nezávislost na implementaci.
Pokud se implementace změní, s velkou pravděpodobností to výsledky testů neovlivní.

\section{Průběžná integrace}

Průběžná integrace (z anglického \textit{Continous integration}, zkráceně \texttt{CI}) je praktika vývoje software, při které členové týmu integrují svou práci mnohdy až několikrát denně.
Každá integrace je automaticky ověřena kompilací na buildovacím serveru a spuštěním testů.
Tato technika si klade za cíl odhalid integrační chyby aplikace co nejdříve je to možné.
To má za následek snížení časové náročnosti implemetace nových funkcí bez rizika narušení funkcionalit původní aplikace.

Nové funckionality se běžně skládají z nového kódu, upraveného produkčního kódu a upravených testů.
V momentě kdy vývojář označí funkcionalitu za kompletní, odešle své změny na vzdálený buildovací server, kde se aplikace zkompiluje a otestuje.
Integrace je následně provedena jen v případě, že celý proces proběhl bez chyby.
Jak je zjevné, při této technice je zásadní vysoké pokrytí aplikace testy.

Průběžné integrování je často velmi blízce vázáno na správu zdrojových kódů.
Některé systémy jako GitHub či GitLab je dokonce možné nastavit tak, aby by změny byly přidány do hlavní větve až ve chvíli, kdy build projde bez komplikací. \cite{travis-ci-building-pr}

Pro průběžnou integraci je na trhu dostupných mnoho řešení.
Rozhodoval jsem se mezi použitím komerčních řešení \textit{Travis CI} a \textit{Circle CI}.
První zmíněná možnost je v open-source komunitě velmi populární a dalo by se říct, že je pro \texttt{CI} téměř synonymem.
Podle nezávislého průzkumu z roku 2016 využívá \textit{Travis CI} téměř pětinásobek uživatelů než ostatních poskytovatelů dohromady. \cite{oregonstate-ci-survey}

Druhé řešení jsem chtěl vyzkoušet kvůli rostoucí popularitě. \cite{circleci-popularity}
Bohužel je kompilace na operačním systému macOS pro open-source projekty možná až po individuálním schválení. \cite{circleci-pricing}
Z tohoto důvodu jsem nakonec zvolil první řešení, kde je kompilace na systému macOS pro studenty zdarma.

\subsection{Statická analýza kódu}

Spolu s roustoucím týmem a rostoucím zdrojovým kódem se často zavádí směrnice programování.
Tyto směrnice udávají jakým způsobem je kód formátovaný.
Při dodržování těchto směrnic se kód stává konzistentním a dobře čitelným, to má za následek zrychlení vývoje a zvyšuje udržitelnost kódu.
Aby nebylo nutné analyzovat kód ručně, existují nástroje které analýzu automatizují.

Pro zajištění konzistence jsem použil nástroj \texttt{SwiftLint} od společnosti \texttt{Provider}.
\texttt{SwiftLint} je nástroj obsahující sadu pravidel pro statickou analýzu kódu.
Mimo jiné kontroluje správné zarovnání kódu, délky řádků či názvy proměnných.
V době psaní práce obsahuje tento nástroj dohromady téměř osmdesát pravidel.
Ve své implemetaci jsem se rozhodl vynechat dvě pravidla, které jsem nepovažoval za stěžejní a jejich dodržování by z mého pohledu vedlo ke snížení čitelnosti.
Kromě těchto dvou pravidel jsem využil i možnost vypnout pravidla pro určité části kódu.
Nejčastěji se jednalo o pravidla na kontrolu délky metod.

Tuto analýzu zajišťuje buildovací server před začátkem kompilace.
Pokud nejsou v kódu závažné porušení směrnice, pokračuje se ke kompilaci.
V opačném případě nastane chyba a je nutné kód opravit a integraci pustit znovu.


\section{Průběžné doručování}

Průběžné doručování (z anglického \textit{Continuous delivery}, zkráceně \textit{CD}) je způsob, kterým lze nasadit nové funkcionality, změny konfigurace či opravy chyb do produkčního prostředí.
Cílem této praktiky je automatizovaně aplikovat opakované postupy potřebné k vydání aplikace.
Tím lze minimalizovat množství chyb, které by vývojář jinak mohl při vydávání způsobit.
Vývojáři stačí v tomto případě pouze doručování spusti, server zařídí aby aplikace byla správně nasazena.

Kromě minimalize výskytu chyb tato technika snižuje čas vývoje.
Díky automatizovanému vydávání mají vývojáři více času na samotný vývoj.
Tím lze zaručit vyšší kvalitu kódu, nižší náklady na vývoj a také častější aktualizace software. \cite{continuousdelivery-what-is-ci}

Přestože pro svou práci nemám reálné produkční prostředí, rozhodl jsem se \textit{CD} použít pro distribuci testovací verze aplikace.
Díky tomu jsem mohl jednotlivé verze vydávat v průběhu vývoje velmi rychle a ověřit funkčnost aplikace na skutečných zařízeních.
Vydání verze jsem nastavil na každé nahrání zdrojových kódů do repozitáře.
Po statické analýze a kompilaci byla aplikace nahrána do prostředí Fabric, odkud si ji mohli registrovaní testeři stáhnout.