\section{Přidání nové tiskárny}

Obrazovka přidání tiskárny slouží pro případ, kdy uživatel chce ovládat novou tiskárnu a nenalezl ji v seznamu síťových tiskáren.
Pro úspěšné přidání tiskárny je nutné zadat její název, URL či IP adresu a přístupový token.
Volitelně je možné také přidat URL adresu na které se vyskytuje video stream z web kamery natáčející průběh tisku.

\subsection{Validace formuláře}

Jako u každé aplikace, která přijímá uživatelské vstupy i v mé implementaci je nutné ověřit platnost zadaných údajů.
Jelikož se jedná o logiku, je validace umístěna ve ViewModelu.
Text zadaný uživatelem se při každé změně odešle ViewModelu, který všechny vstupy zvaliduje.
Na základě platnosti kombinace vstupů vyšle ViewModel boolean hodnotu Controlleru.
Controller pomocí UI bindigs sváže tuto hodnotu s blokací tlačítka \uv{Přihlásit}. 

Je-li formulář platný, tlačítko je povolené.
V opačném případě je zablokované a nelze stisknutím spustit jeho interakci.

Hodnoty formuláře jsou validované pomocí operací nad signály.
Signály vstupů jsou nejprve zkombinovány a následně map operátorem převedeny na booleanovskou hodnotu.
Použití validace nad signály je vidět v ukázce \ref{code:add-printer-validation}.

\swiftcode{code:add-printer-validation}{Validace formuláře pomocí signálů}{assets/code/add-printer-validation.swift}

