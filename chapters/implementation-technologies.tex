\section{Použité technologie}

V této části se postupně zabývám jednotlivými body analýzy a rozebírám jaké řešení z těch, která jsem analyzoval, jsem zvolil pro implementaci.
Jedním z hlavních kritérií při konečném rozhodování byla slučitelnost jednotlivých částí.
Vzhledem k rozsahu implementace jsem si musel být jistý, že jednotlivé části budou fungovat správně v celé aplikaci.
Rozhodujícím faktorem tedy nebyla popularita jednotlivých řešení ale spíše jejich flexibilita.

\subsection{Architektura aplikace}

Nejdůležitějším rozhodnutím bylo správné vybrání architektury.
Špatný výběr architektury by mohl mít dopad na celkovou funkčnost aplikace či znemožnit implementaci některých z vyžadovaných funkcí.

Po podrobném analyzování obou zmíněných architektur jsem se rozhodl využít MVVM.
Přestože MVVM není v současné době tak rozšířené, nabízí spoustu vlastností, které standardní MVC nemá.

Silným argumentem je testování.
V architektuře MVVM lze jednotlivé části otestovat samostatně.
K otestování logiky aplikace navíc není potřeba vizuální vrstva aplikace.

Neméně podstatným argumentem je čitelnost kódu.
Vrstva View Controller obsahuje mnohem méně kódu a lze se v něm snáze orientovat.
ViewModel naopak neobsahuje žádné prvky uživatelského rozhraní a znázorňuje tak pouze způsob jakým daná část aplikace chová.
Jednotlivé vrstvy jsou od sebe striktně odděleny a dodržují princip jedné odpovědnosti \cite{toptal-srp}.

\subsection{Synchronizace vláken}

Pro synchronizaci vláken jsem se rozhodl použít knihovny ReactiveCocoa a ReactiveSwift, tedy reaktivní přístup.
Díky těmto knihovnám lze jednoduše tvořit závislosti asynchronních operací.

S využitím knihovny ReactiveCocoa od verze 5.0 lze navíc využít tkzv. \textit{UI bindings}.
Ty zaručují, že hodnoty signálů jsou vždy zpracovány na hlavním vlákně a to i v momentě, kdy jsou odeslány z vlákna v pozadí.
ReactiveSwift pak zaručuje konzistency dat mezi vlákny.
Dohromady tak tyto knihovny zamezují vzniku \textit{race condition} za běhu aplikace.
Více o těchto knihovnách lze zjistit z oficiální dokumentace \cite{github-reactiveswift} a \cite{github-reactivecocoa}.

Neméně podstatným faktorem byla i velmi dobrá integrace v architektuře MVVM.
Pomocí operátorů nad signály lze formátovat vlastnosti modelových objektů (přidání jednotek, standardizace čísel, zástupné texty).
Takto naformátované signály zaručí, že kdykoliv se aktualizuje modelový objekt, jeho vlastnosti budou správně naformátovány.
ViewController tak vždy dostane data ve správném formátu bez ohledu na to, jakým způsobem byly změny na modelu aplikovány.

\subsection{Síťová vrstva}

Síťovou vrstvu jsem se zpočátku rozhodl implementovat knihovnou Alamofire.
Důvodem byla snazší implementace a integrace do aplikace.
Protože nemá Alamofire striktně danou strukturu koncových bodů, bylo jednoduché implementovat dynamickou URL tiskárny.

Chybějící podpora reaktivního programování ale velmi brzy negativně zasáhla běh aplikace.
K synchronizaci vláken jsem nemohl využít ReactiveSwift a aplikace byla zatížena množstvím chyb vznikajících při vykonávání síťových požadavků a následné serializaci dat.

Z tohoto důvodu jsem se rozhodl využít knihovny Moya, která nabízí reaktivní rozšíření a je kompatibilní s ReactiveSwift.

Veškeré síťové požadavky nakonec využívají knihovnu Moya, s jejíž implementací chyby s vlákny vymizely.
