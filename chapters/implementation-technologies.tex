\section{Použité technologie}

V této části se postupně zabývám jednotlivými body analýzy a rozebírám jaké řešení z těch, která jsem analyzoval, jsem zvolil pro implementaci.
Jedním z hlavních kritérií při konečném rozhodování byla slučitelnost jednotlivých částí.
Vzhledem k rozsahu implementace jsem si musel být jistý, že jednotlivé části budou fungovat správně v celé aplikaci.
Rozhodujícím faktorem tedy nebyla popularita jednotlivých řešení ale spíše jejich flexibilita.

\subsection{Architektura aplikace}

Nejdůležitějším rozhodnutím bylo správné vybrání architektury.
Špatný výběr architektury by mohl mít dopad na celkovou funkčnost aplikace či znemožnit implementaci některých z vyžadovaných funkcí.

Po podrobném analyzování obou zmíněných architektur jsem se rozhodl využít MVVM.
Přestože MVVM není v současné době tak rozšířené, nabízí spoustu vlastností, které standardní MVC nemá.

Silným argumentem je testování.
V architektuře MVVM lze jednotlivé části otestovat samostatně.
K otestování logiky aplikace navíc není potřeba vizuální vrstva aplikace.

Neméně podstatným argumentem je čitelnost kódu.
Vrstva View Controller obsahuje mnohem méně kódu a lze se v něm snáze orientovat.
ViewModel naopak neobsahuje žádné prvky uživatelského rozhraní a znázorňuje tak pouze způsob jakým daná část aplikace chová.
Jednotlivé vrstvy jsou od sebe striktně odděleny a dodržují princip jedné odpovědnosti \cite{toptal-srp}.