\section{Průběžné doručování}

Průběžné doručování (z anglického \textit{Continuous delivery}, zkráceně \textit{CD}) je způsob, kterým lze nasadit nové funkcionality, změny konfigurace či opravy chyb do produkčního prostředí.
Cílem této praktiky je automatizovaně aplikovat opakované postupy potřebné k vydání aplikace.
Tím lze minimalizovat množství chyb, které by vývojář jinak mohl při vydávání způsobit.
Vývojáři stačí v tomto případě pouze doručování spusti, server zařídí aby aplikace byla správně nasazena.

Kromě minimalize výskytu chyb tato technika snižuje čas vývoje.
Díky automatizovanému vydávání mají vývojáři více času na samotný vývoj.
Tím lze zaručit vyšší kvalitu kódu, nižší náklady na vývoj a také častější aktualizace software. \cite{continuousdelivery-what-is-ci}

Přestože pro svou práci nemám reálné produkční prostředí, rozhodl jsem se \textit{CD} použít pro distribuci testovací verze aplikace.
Díky tomu jsem mohl jednotlivé verze vydávat v průběhu vývoje velmi rychle a ověřit funkčnost aplikace na skutečných zařízeních.
Vydání verze jsem nastavil na každé nahrání zdrojových kódů do repozitáře.
Po statické analýze a kompilaci byla aplikace nahrána do prostředí Fabric, odkud si ji mohli registrovaní testeři stáhnout.