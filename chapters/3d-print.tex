\chapter{3D tisk}\label{3d-tisk}

V této kapitole stručně zmiňuji princip a technologii 3D tisku.
Čtenáře uvedu také do stávajícího stavu z pohledu ovládání 3D tiskáren a využití mobilních aplikací k tomuto účelu.

\section{Technologie 3D tisku}\label{3d-tisk-technologie}

3D tisk je proces, při kterém lze z digitálního modelu vytvořit reálný objekt.
Tyto objekty vznikají postup nanášením jednotlivých vrstev modelu na sebe ve vertikálním směru.
Jedna z metod využívajících vrstvení materiálu se nazývá Fused deposition modeling (zkráceně \acrshort{fdm}).
Princip \acrshort{fdm} spočívá v tavení plastu (případně jiných materiálů) pomocí tiskové hlavy, která následně nanáší taveninu ve vrstvách na sebe \cite{3d-print-fdm}.
Takto vytvořené plastové objekty mají velkou výhodu v nízkých nákladech na výrobu.
Jsou tedy ideálním prostředkem k výrobě prototypů nebo produkování omezeného množství výrobků \cite{3d-print-for-prototyping}.

\section{OctoPrint - Ovládání 3D tiskárny}\label{3d-tisk-ovladani}

Tisk lze obsloužit pomocí mnoha aplikačních rozhraní.
Velmi oblíbeným nástrojem je OctoPrint, který nabízí ovládání pomocí webového prohlížeče.
Jedná se tedy o webové rozhraní, které reprezentuje uživateli příkazy tiskárny pomocí tlačítek či jiných grafických prvků.
Z prohlížeče je tak možné měnit základní vlastnosti jako např. teploty, průběh tisku či nastavení připojení k tiskárně.
Bohužel je ale toto rozhraní zcela zaměřeno na použití z počítače.
Ovládací prvky nejsou dostatečně veliké a manipulace s nimi přináší velmi špatnou uživatelskou zkušenost na mobilních zařízeních.
Možné alternativy v kapitole \textit{Analýza}.

Kromě webového rozhraní nabízí OctoPrint také \acrshort{api} pro aplikace třetích stran.
Pro implementaci své aplikace využiji právě tohoto rozhraní.

V době psaní práce je dostupná verze \acrshort{api} $1.3$, která oproti předchozí verzi nabízí nové funkce jako např. správu uživatelů či nastavení systému.
Ty ale sám autor označuje za experimentální.
Pro mobilní aplikaci navíc nejsou stěžejní.
Z tohoto důvodu jsem se rozhodl využít starší verzi $1.2.15$, která většinu funkcí označuje jako stabilní.

\section{Definice pojmů}\label{3d-tisk-definice-pojmu}

Přestože se obvykle slovem tisk rozumí běžný papírový tisk pomocí inkoustové nebo laserové tiskárny,
ve své práci tento druh tisku vůbec neuvažuji a pojmem tisk rozumím 3D tisk.
Analogické předefinování si dovolím i u pojmů odvozených.

Ve své práci uvažuji jako metodu tisku pouze \acrshort{fdm} zmíněnou v kapitole \textit{Technologie 3D tisku}.
