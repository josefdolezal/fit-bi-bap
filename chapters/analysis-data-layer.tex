\section{Datová vrstva}\label{analyza-datova-vrstva}

Pro uložení nastavení tiskáren je důležité, aby aplikace mohla ukládat data na disk.
Většina ostatních dat není potřeba ukládat, protože je velmi pravděpodobné, že při dalším spuštění aplikace už by data nebyla aktuální.
I přesto jsem se rozhodl veškerá data, která aplikace získá z tiskárny uložit lokálně.
Lokální uložiště následně bude sloužit jako \textit{source of truth}, tedy kdykoliv vznikne v aplikaci nekonzistence, data se obnoví z lokálního uložiště.

Pro lokální uložiště jsem vybral dvě technologie, které porovnám a následně jednu implementuji.
První technologií jsou \textit{Core Data}, technologie vyvinutá společností Apple běžně používaná na platformách iOS a macOS.
Druhou technologií je Realm Mobile Database, open source knihovna třetí strany.

\subsection{Core Data}

Core Data je framework představený společností Apple v roce 2005 \cite{objcio-core-data}.
Jedná se o technologii starající se o modelovou vrstvu aplikace.
Kromě reprezentování stavu aplikace umožňují Core Data také data perzistovat na disk.
Nejedná se ale pouze o \acrshort{orm} či \acrshort{sql} wrapper, Core Data spravuje data i ve chvíli, kdy jsou v paměti.

Během analyzování tohoto frameworku jsem dospěl k závěru, že implementace by byla velmi složitá.
Z tohoto důvodu jsem neprováděl hlubší analýzu.

\subsection{Realm Mobile Database}\label{datova-vrstva-realm}

Jako druhou možnost jsem zvolil technologii Realm.
Přestože verze 1.0 této knihovny byla vydána teprve v květnu 2016, má k dnešnímu dni má dohromady více než dvacet tisíc hvězdiček na službě Github \cite{github-realm-repos}.

Realm je modelová technologie vytvořená pro využití v mobilních zařízeních.
Oproti běžným databázím Realm nevyužívá \acrshort{orm} a nekopíruje data z databáze.
V Realm aplikace pracuje s reálnými daty uloženými na disku, z tohoto důvodu jakmile jsou data jednou aktualizovány, změna se automaticky propisuje všude, kde jsou data využívána (reaktivní přístup) \cite{realm-overview}.

\subsubsection*{Modelové objekty}

Realm je založen na modelových objektech.
Ty se definují v jazyku, ve kterém je aplikace programována (v mém případě Swift).
Databázový soubor se následně na disku vytvoří z těchto modelových objektů.
Přístup je tedy opačný než u \acrshort{orm}, kde se v aplikaci vytváří objekty podle toho, jak jsou uloženy v databázi.
Jedinou podmínkou při vytváření objektu je nutnost dědit od třídy \textit{Object} a využít správné datové typy (ne všechny je možné serializovat na disk). V ukázce \ref{code:realm-object} je třída \textit{Printer}, kterou lze použít pro reprezentaci objektu tiskárny.

\swiftcode{code:realm-object}{Modelový objekt v technologii Realm}{assets/code/realm-object.swift}

\subsubsection*{Relace}

Realm implementačně sice není relační databází, nabízí ale podobné \acrshort{api} pro relace mezi objekty.
Objekty mohou být v relacích \textit{many-to-many}, \textit{one-to-many}, \textit{ono-to-one} ale i \textit{many-to-one}.
Na základě těchto vlastností je následně možné vytvářet databázové dotazy.

\subsubsection*{Transakce}

Z důvodu zajištění konzistence dat v aplikaci je nutné modelové objekty upravovat v transakcích \cite{realm-swift}.
Realm neodděluje standardní \textit{\acrshort{crud}} operace, místo toho je zavedena \textit{write} transakce, v níž je možné CRUD operace využít. Transakce implementuji v ukázce \ref{code:realm-write}.

\swiftcode{code:realm-write}{Transakce v technologii Realm}{assets/code/realm-write.swift}

\subsubsection*{Notifikace}

Velkou výhodou této technologie jsou notifikace.
Realm umožňuje notifikovat o změnách objektů v databázi.
Na základě těchto notifikací pak lze např. překreslit obrazovku či změnit stav aplikace.

Notifikace lze zaregistrovat i na změny kolekcí.
Zobrazuje-li např. aplikace seznam tiskáren, lze se zaregistrovat na jejich změny.
Je-li pak do kolekce přidána položka nebo je stávající položka změněna, lze nechat překreslit seznam.

Tomuto principu se říká \textit{Reaktivní programování} a lze ho velmi dobře integrovat do architektury \acrshort{mvvm}.
