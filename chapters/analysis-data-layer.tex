\section{Datová vrstva}\label{analyza-datova-vrstva}

Pro uložení nastavení tiskáren je důležité, aby aplikace mohla ukládat data na disk.
Většina ostatních dat není potřeba ukládat, protože je velmi pravděpodobné, že při dalším spuštění aplikace už by data nebyla aktuální.
I přesto jsem se rozhodl veškerá data, která aplikace získá z tiskárny uložit lokálně.
Lokální uložiště následně bude sloužit jako \textit{source of truth}, tedy kdykoliv vznikne v aplikaci nekonzistence, data se obnoví z lokálního uložiště.

Pro lokální uložiště jsem vybral dvě technologie, které porovnám a následně jednu implementuji.
První technologií jsou \textit{Core Data}, technologie vyvinutá společností Apple běžně používaná na platformách iOS a macOS.
Druhou technologií je Realm Mobile Database, open source knihovna třetí strany.

\subsection{Core Data}

Core Data je framework představený společností Apple v roce 2005.
Jedná se technologii starající se o modelovou vrstvu aplikace.
Kromě reprezentování stavu aplikace umožňují Core Data také data perzistovat na disk.
Nejedná se ale pouze o ORM či SQL wrapper, Core Data spravuje i ve chvíli, kdy jsou v paměti.

Během analyzování tohoto frameworku jsem dospěl k závěru, že implementace by byla velmi složitá.
Z tohoto důvodu jsem neprováděl hlubší analýzu.
