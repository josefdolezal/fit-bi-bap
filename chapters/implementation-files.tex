\section{Správa souborového systému}

Druhou obrazovkou detailu tiskárny je správa souborů.
Uživatel vidí seznam souborů a může je třídit.

Při příchodu na obrazovku se vyšle požadavek na OctoPrint pro stažení souborů.
\texttt{ViewModel} opět dbá pouze na data uložená v lokální databázi.
Jakmile tedy přijde odpověď na požadavek, data se uloží do databáze a informace se pomocí signálu propaguje do \texttt{Controlleru}.

\subsection{Filtrování}

Ve \texttt{ViewModelu} jsem připravil tři filtry, ty představují \uv{vše}, \uv{soubory na tiskárně} a \uv{soubory na paměťové kartě}.
Implemetace filtrů je vidět v ukázce \ref{code:files-filters}.
Výchozí filtr zobrazuje soubory dostupné jak v paměti tiskárny tak na paměťové kartě.
Jakmile se rozhodne uživatel změnit filtr, vytvoří svou interakcí vstup do \texttt{ViewModelu} kde se jen změní aktivní filtr a zpět se upozorní \texttt{Controller}.

\swiftcode{code:files-filters}{Filtry souborů}{assets/code/files-filters.swift}

Tímto jsem zajistil, že nevznikne nekonzistence při aktualizaci souborů.
Tedy vždy jsou zobrazeny právě ty soubory, které uživatel zvolil.
Žádné nechybí ani nepřebývají ani v době, kdy se seznam znovu stáhne z OctoPrint.

\subsection{Nahrávání souborů}

Nahrávání souborů k tisku je z mého pohledu důležitou funkcí této aplikace.
Systém iOS je ale bohužel uzavřený a nedovoluje uživateli přistupovat k souborovému systému \cite{imobile-access-fs}.
Umožňuje ale otevírat soubory z jiných aplikací.
Tyto aplikace musí funkci poskytování souborů explicitně povolit.

Při implementaci této funkce stačilo v nastavení projektu zvolit jaký typ souborů aplikace umí otevřít.
Systém se následně postará o zobrazení dialogu uživateli a provede ho výběrem souboru.
Jakmile uživatel soubor vybere, systém ho předá aplikaci.

\image{ref:implementation-utis}{Nastavení projektu pro otevírání souborů}{assets/implementation-utis.png}

\bigskip{}

Po výběru se předá umístění souboru do \texttt{ViewModel} vrstvy, která pomocí knihovny \texttt{Moya} soubor do tiskárny nahraje.

Standardně má uživatel jako jediné dostupné uložiště iCloud implementovaný přímo v systému, používá-li ale některé z cloudových uložišť může soubor nahrát přímo z nich.
