\section{Správa souborového systému}

Druhou obrazovkou detailu tiskárny je správa souborů.
Uživatel vidí seznam souborů a může je třídit.

Při příchodu na obrazovku se vyšle požadavek na OctoPrint pro stažení souborů.
ViewModel opět dbá pouze na data uložená v lokální databázi.
Jakmile tedy přijde odpověď na požadavek, data se uloží do databáze a informace se pomocí signálu propaguje do Controlleru.

\subsection{Filtrování}

Ve ViewModelu jsem připravil tři filtry, ty představují \uv{vše}, \uv{soubory na tiskárně} a \uv{soubory na paměťové kartě}.
Implemetace filtrů je vidět v ukázce \ref{code:files-filters}.
Výchozí filtr zobrazuje soubory dostupné jak v paměti tiskárny tak na paměťové kartě.
Jakmile se rozhodne uživatel změnit filtr, vytvoří svou interakcí vstup do ViewModelu kde se jen změní aktivní filtr a zpět se upozorní Controller.

\swiftcode{code:files-filters}{Filtry souborů}{assets/code/files-filters.swift}

Tímto jsem zajistil, že nevznikne nekonzistence při aktualizaci souborů.
Tedy vždy jsou zobrazeny právě ty soubory, které uživatel zvolil.
Žádné nechybí ani nepřebývají ani v době, kdy se seznam znovu stáhne z OctoPrint.
