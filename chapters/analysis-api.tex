\section{OctoPrint API}

V implementační části práce se zabývám vytvořením mobilní aplikace pro OctoPrint.
Pro pro olvádání tiskárny využiji REST API, které OctoPrint v základu nabízí.
V dalších sekcích se budu podrobně zabývat analýzou jednotlivých funkcionalit implementovaných v mém řešení.

Z důvodu bezpečnosti musí být každý požadavek na tiskárnu autorizovaný.
To OctoPrint zajišťuje pomocí ověřovacích \textit{tokenů}.
Token je náhodně vygenerovaná posloupnost znaků jednoznačně spjatá s účtem uživatele.
Pokud v požadavku je uveden platný token, předpokládá se, že požadavek vytvořil uživatel jehož účet je s token svázaný.

\subsection{Verze API a ověřování}

První analyzovanou funkcionalitou je verze API.
Tento endpoint je dostupný na \inlinestr{api/version}.
V případě platného přístupového tokenu vrátí textovou reprezentaci verze API, v opačném případě odpovídající stavový kód.

Ve své implementaci jsem se rozhodl uživatele nezatěžovat využívanými verzemi.
Jedná se ale o jediný endpoint, který nemá žádné vedlejší efekty.
Zároveň neexistuje žádný endpoint umožňující samostatně ověřit platnost tokenu.

Vzhledem k výše uvedenému je tato funkcionalita vhodná pro simulaci přihlašování k tiskárně.

\subsection{Správa souborového systému}

Jednou ze skupin funkcionalit REST API je správa souborového systému.
Pomocí API lze nejen obsah systému číst, ale také s ním manipulovat.
Jednotlivé endpointy jsou zanalyzovány níže.

\subsubsection*{Seznam souborů}

OctoPrint umožňuje vypsat seznam uložených souborů.
Tento endpoint je dostupný na \inlinestr{api/files}.

V současnou chvíli mohou na soborový systém být uloženy pouze tisknutelné soubory (s příponou .gcode) nebo modelové soubory (s příponou .stl).

Uloženy mohou být přímo na zařízení kde je OctoPrint nainstalovaný (označované jako lokální) nebo na paměťovou kartu.
Výše zmíněná funkce stáhne seznam všech souborů bez ohledu na jejich lokace.
Příznak lokace je v odpovědi uveden u každého souboru zvlášť.

Lze ale stáhnout soubory z obou lokalit samostatně.
Pro stažení seznamu lokálních souborů existuje koncový bod \inlinestr{api/files/local}.
Seznam ouborů uložených na paměťové kartě je možné stáhnout z \inlinestr{api/files/sdcard}.

O souborech jsou dostupná jejich metadata (název, velikost, datum změny).
GCode může navíc obsahovat analýzu tisku (předpokládaná doba tisku, spotřeba filamentu a statistiky tisknutí).

Struktura odpovědí je pro všechny tři požadavky stejná, liší se pouze obsah seznamu v závislosti na vybrané lokalitě.

\subsubsection*{Nahrávání souborů}

Pomocí API je možné nejen číst informace o existujících souborech, ale také nahrávat nové.
Pro nahrávání slouží endpointy \inlinestr{api/files/local} a \inlinestr{api/files/sdcard}.
Podle zvoleného koncového bodu se soubor uloží buď přímo na zařízení nebo na paměťovou kartu.

Soubor se odesílá v těle požadavku kde je zároveň uveden jeho název.
V případě vzniku konfliktu názvů bude stávající soubor nahrazen novým.

Systém iOS neobsahuje sdílený souborový systém, soubory lze nahrávat buď pouze z aplikace nebo lze využít cloudových služeb \cite{apple-file-system-basics}.

\subsubsection*{Informace o konkrétním souboru}

Aby bylo možné aktualizovat informace o jednotlivých souborech, umožňuje OctoPrint využít koncový bod \inlinestr{api/files/[location]/[filename]}.
Takový požadavek zajistí stažení aktuálních informací o souboru v lokaci \textit{[location]} s názvem \textit{[filename]}.

Strukturou se odpověď nelyší od položky souboru v seznamu souborů.
Požadavek lze ale využít v momentě, kdy se z klientské aplikace se souborem v tiskárně manipuluje pro udržení konzistence.

\subsubsection*{Manipulace se soubory -- Tisk}

OctoPrint umožňuje pro manipulaci se soubory více konfigurací.
Pro využití v mobilní aplikaci se ale hodí pouze konfigurace pro tisk.
Ta je dostupná na endpointu \inlinestr{api/files/[location]/[filename]}.

V těle požadavku je nutné uvést typ příkazu a explicitně vyžádat tisknutí.
Více o konfiguracích se lze dočíst v dokumentaci \cite{octoprint-api-filecommand}.

\subsubsection*{Smazání souboru}

Další funkcionalitou je odstranění souboru.
Tento endpoint je dostupný na \inlinestr{api/files/[location]/[filename]}.
Požadavek je nutné odeslat metodou \textit{DELETE}.

\subsection{Připojení tiskárny}

Protože OctoPrint běží na samostatném zařízení a hardware tiskárny jen ovládá, je nutné ho k tiskárně explicitně připojit.
Připojení probíhá pomocí výběru dostupného portu.
Seznam dostupných portů poskytuje sám OctoPrint.
