\section{OctoPrint API}

V implementační části práce se zabývám vytvořením mobilní aplikace pro OctoPrint.
Pro pro olvádání tiskárny využiji REST API, které OctoPrint v základu nabízí.
V dalších sekcích se budu podrobně zabývat analýzou jednotlivých funkcionalit implementovaných v mém řešení.

Z důvodu bezpečnosti musí být každý požadavek na tiskárnu autorizovaný.
To OctoPrint zajišťuje pomocí ověřovacích \textit{tokenů}.
Token je náhodně vygenerovaná posloupnost znaků jednoznačně spjatá s účtem uživatele.
Pokud v požadavku je uveden platný token, předpokládá se, že požadavek vytvořil uživatel jehož účet je s token svázaný.

\subsection{Verze API a ověřování}

První analyzovanou funkcionalitou je verze API.
Tento endpoint je dostupný na \inlinestr{api/version}.
V případě platného přístupového tokenu vrátí textovou reprezentaci verze API, v opačném případě odpovídající stavový kód.

Ve své implementaci jsem se rozhodl uživatele nezatěžovat využívanými verzemi.
Jedná se ale o jediný endpoint, který nemá žádné vedlejší efekty.
Zároveň neexistuje žádný endpoint umožňující samostatně ověřit platnost tokenu.

Vzhledem k výše uvedenému je tato funkcionalita vhodná pro simulaci přihlašování k tiskárně.