\section{Terminálový emulátor}

Abych uživateli umožnil vytvořit libovolný příkaz tiskárně, vytvořil jsem obrazovku připomínající terminálové rozhraní.
Obrazovka se skládá z historie příkazů a vstupu pro nový příkaz.

\subsection{Vytvoření příkazu}

Jakmile uživatel napíše příkaz a stiskne tlačítko \uv{Odeslat}, vytvoří ViewModel na základě vstupu novou položku v lokální databázi.
Příkaz je v databázi vytvořen s příznakem \uv{Zpracováván}.
Na základě vložení ViewModel upozorní \texttt{Controller}, že v historii příkazů proběhla změna a \texttt{Controller} překreslí seznam.

V rámci dodržení architektury \acrshort{mvvm} jsem pro každou položku seznamu vytvořil další ViewModel.
ViewModel položky seznamu se stará o to, aby byl příkaz proveden.
Ve chvíli vytvoření položky v seznamu ViewModel prozkoumá objekt příkazu.
Má-li příznak \uv{Zpracováván}, vytvoří požadavek na tiskárnu o provedení příkazu.
Na základě odpovědi je nastaven nový příznak.
V případě úspěchu \uv{Zpracován}, v opačném případě \uv{Chyba}.

ViewModel o výsledku informuje položku seznamu (View).
Pokud je výsledkem chyba, View se červeně podbarví aby uživateli naznačilo, že příkaz selhal.
