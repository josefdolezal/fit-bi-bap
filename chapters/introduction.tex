V posledních letech můžeme být svědky velkého technologického rozmachu.
Nedílnou součástí tohoto rozmachu jsou mobilní aplikace a také 3D tiskárny.
V oblasti 3D tiskáren se popularita zvýšila za posledních pět let více než osmkrát (měřeno dle zájmu o téma v Google vyhledávání) \cite{3d-print-google-trends}.
Mnohem většího úspěchu dosáhly mobilní zařízení, která v září 2016 poprvé přesáhly využívání osobních počítačů \cite{mobile-devices-market-share}.

I navzdory těmto faktům je v současné době využívání a ovládání 3D tiskáren doménou zejména počítačů.
O tom svědčí především rozšíření mobilních aplikací.
Ty jsou v současné době pouze dvě, jejich funkcionalita je navíc oproti desktopové verzi značně omezená.

Ve své bakalářské práci se zabývám návrhem mobilní aplikace pro ovládání 3D tiskáren prostřednictví rozhraní OctoPrint, její analýzou a implementací.
Práci uvádím kapitolou \textit{3D tisk}, která čtenáře stručně uvede do problému.
V kapitole \textit{Analýza} zkoumám možné technologie a paradigmata.
Použité technologie následně zmiňuji v kapitole o implementaci.
V části věnující se implementaci popisuji jaké funkcionality aplikace obsahuje a způsob jakým jsem je do aplikace zakomponoval.
O udržení kvality aplikace a využití automatizace v průběhu vývoje mluvím v kapitole \textit{Testování}.

Výsledná aplikace je vedena jakou open source.
Její zdrojový kód je tedy volně dostupný s možností libovolných úprav.
Přestože aplikace není distribuována standardní cestou obchodem App Store, je na tuto formu distribuce plně připravena.
Jednotilivé verze aplikace jsou v tuto chvíli distribuovány pouze registrovaným vývojářům pomocí Fabric Beta.\footnote{Komerční platforma umožňující distribuci aplikace vývojářům mimo služby Apple}
