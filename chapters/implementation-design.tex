\section{Návrh vzhledu}

Po úplném zanalyzování problému je přistoupil k návrhu aplikace.
Uživatelské rozhraní jsem se snažil tvořit jednoduché a intuitivní.

Návrh jsem vytvářel z kostry aplikace.
Nejdříve jsem ze seznamu funkcionalit vybral takové, které se budou nacházet na jedné obrazovce.
Následně jsem vytvořil mapu obrazovek.
Část mapy lze vidět na obrázku \ref{fig:implementation-design}.

\image{fig:implementation-design}{Mapa obrazovek}{assets/implementation-design.png}

V mapě obrazovek jsem také znázornil průchod aplikací a u některých funkcionalit vytvořil wireframe či výčet funkcí.
V návrhu jsem se dále nepouštěl do kompletních grafických zpracování.
Vzhledem obrazovek a jejich rozložením jsem se více zabýval až při konkrétním návrhu.

Obecným požadavkem byl univerzální vzhled.
iOS umožňuje vytvářet univerzální aplikace, které fungují s jedním vzhledem na iPhone i iPad.
Design se programuje relativně, tím je zajištěno, že funguje i na větších zařízeních. \cite{designcode-ios-guidelines}
Tímto způsobem jsem optimalizoval aplikaci pro obě zařízení.
