\section{Detail a přehled tiskárny}

V momentě, kdy má uživatel přidanou tiskárnu s platným přístupovým tokenem, může se jejím výběrem ze seznamu dostupných tiskáren dostat na detail.
Obrazovka detailu se skládá ze tří hlavních záložek, z nichž každá pak v rámci svých funkcionalit prezentuje další obrazovky.

První částí, která se uživateli zobrazí je přehled tiskárny.
Přehled je klíčovou funkcí aplikace.
Umožňuje totiž velmi rychle získat základní údaje o probíhajícím tisku.

\subsection{Připojení tiskárny}

OctoPrint se může nacházet ve stavu, kdy je připravený k použití, ale není připojen k tiskárně.
V tuto chvíli nelze získat informace o aktuálním tisku ale ani o tiskárně samotné.

Z tohoto důvodu jsem implementoval \uv{prázdnou obrazovku}, která uživatele upozorňuje na odpojenou tiskárnu a vyzývá uživatele k jejímu připojení.
Fakt, že data nejsou dostupná se propaguje z ViewModelu.
Ten vysílá signál nazvaný \textit{contentIsAvailable} booleanovského typu.
Obrazovka s výzvou k připojení tiskárny je dostupná právě když hodnota signálu je \textit{false}.

Pokud se uživatel rozhodne tiskárnu připojit, může k tomu využít tlačítko \uv{Připojit}.
To zařídí otevření nové obrazovky.
Na nové obrazovce může uživatel ze seznamu dostupných portů vybrat, k jaké tiskárně se má OctoPrint připojit.
Běžně tento seznam bude obsahovat právě jednu položku, protože každou tiskárnu obsluhuje právě jedna instance OctoPrint.
