\section{Detail a přehled tiskárny}

V momentě, kdy má uživatel přidanou tiskárnu s platným přístupovým tokenem, může se jejím výběrem ze seznamu dostupných tiskáren dostat na detail.
Obrazovka detailu se skládá ze tří hlavních záložek, z nichž každá pak v rámci svých funkcionalit prezentuje další obrazovky.

První částí, která se uživateli zobrazí je přehled tiskárny.
Přehled je klíčovou funkcí aplikace.
Umožňuje totiž velmi rychle získat základní údaje o probíhajícím tisku.

\subsection{Připojení tiskárny}

OctoPrint se může nacházet ve stavu, kdy je připravený k použití, ale není připojen k tiskárně.
V tuto chvíli nelze získat informace o aktuálním tisku ale ani o tiskárně samotné.

Z tohoto důvodu jsem implementoval \uv{prázdnou obrazovku}, která uživatele upozorňuje na odpojenou tiskárnu a vyzývá ho k jejímu připojení.
Fakt, že data nejsou dostupná se propaguje z \texttt{ViewModelu}.
Ten vysílá signál nazvaný \texttt{contentIsAvailable} booleanovského typu.
Obrazovka s výzvou k připojení tiskárny je dostupná právě když hodnota signálu je \texttt{false}.

Pokud se uživatel rozhodne tiskárnu připojit, může k tomu využít tlačítko \uv{Připojit}.
To zařídí otevření nové obrazovky.
Na nové obrazovce může uživatel ze seznamu dostupných portů vybrat, k jaké tiskárně se má OctoPrint připojit.
Běžně tento seznam bude obsahovat právě jednu položku, protože každou tiskárnu obsluhuje právě jedna instance OctoPrint.

\subsection{Aktuální tisk}\label{detail-tiskarny-aktualni-tisk}

V momentě kdy je tiskárna v pořádku připojena k OctoPrint rozhraní, zobrazuje aplikace informace o jejím stavu.
Součástí těchto informací jsou teploty podložky a trysky, aktuální tisk a video stream.

Protože všechny tyto informace nejsou dostupné z jediného koncového bodu, aplikace požadavky řetězí.
Postupně jsou volány požadavky pro aktuální stav tiskárny, aktuální tisk, informace o trysce a informace o podložce.
Implementace řetězení je vidět v ukázce \ref{code:detail-requests-chain}.

\swiftcode{code:detail-requests-chain}{Řetězení požadavků v Moya}{assets/code/detail-requests-chain.swift}

Řetězení pomocí reaktivního přístupu přináší možnost chyby řešit pomocí \texttt{railway} strategie.
Tedy řetězení probíhá po \uv{úspěšné koleji} až do výskytu první chyby.
Chyba slouží jako výhybka z této koleje a dále pokračuje po \uv{chybné koleji}.
Tento přístup umožňuje nezabývat se v jednotlivých požadavcích chybami.
Jakmile se v jednom požadavku vyskytne chyba, další požadavky se nevykonají.
Chybu tedy stačí ošetřit na jednom místě.
Více informací o této strategii je dostupné na \cite{fsharp-railway-strategy}.

Video stream aktuální tisku tisku je implementovaný pomocí pravidelného stahování obrázku tisknutí.
Aby aplikace nezatěžovala zbytečně síťové rozhraní, obrázek stahuje nejdříve pět vteřin po poslední aktualizaci.

\subsection{Využití síťových socketů}

V sekci \ref{detail-tiskarny-aktualni-tisk} se zmiňuji o zobrazení informací o aktuálním tisku.
Ty se stahují při příchodu na obrazovku a následně se zobrazují uživateli.
Tisk je ale kontinuální činnost a informace o něm je potřeba aktualizovat.
Interval aktualizování dat by měl být co nejkratší, aby informace co nejpřesněji vystihovaly skutečnost.

Pomocí běžných síťových požadavků by ale mohlo dojít k větší zátěži sítě a rychlejšímu vybíjení mobilního zařízení.
OctoPrint proto poskytuje rozhraní pro komunikaci skrz sockety.

Toto rozhraní umožňuje vytvořit jedno spojení s OctoPrint a pomocí něj dostávat průběžně nové informace.
Na straně OctoPrint je rozhraní implementováno knihovnou \texttt{SockJS}.

Komunikace funguje tak, že klientské zařízení otevře nové spojení na adrese \inlinestr{sockjs/[id1]/[id2]/xhr\_streaming}.
Parametry \texttt{id1} a \texttt{id2} mohou být voleny náhodně.

Po navázání komunikace OctoPrint průběžně odesílá informace klientskému zařízení.
V těchto infomarcích se také vyskytuje \acrshort{json} obejkt s aktuálním stavem tiskárny.

Ve své implementaci jsem chtěl využít knihovny \texttt{Moya}, kterou využívám ke všem ostatním síťovým požadavkům.
V současné době ale nepodporuje tento typ komunikace.
Musel jsem tedy využít zmíněné knihovny \texttt{Alamofire}, kterou \texttt{Moya} využívá ve své implementaci.

\subsubsection*{Reaktivní rozšíření}

Abych dosáhl reaktivního přístupu k tomuto typu komunikace, vytvořil jsem pro \texttt{Alamofire} obalovou třídu \texttt{WebSocket}.
Jejím jediným účelem je vytvořit síťové spojení s libovolným síťovým zařízením, které tento typ komunikace podporuje.

Pro vytvoření spojení stačí využít metodu \texttt{connect(url:\_, method:\_)}, která jako výsledek vrátí \texttt{Cold} signál.
Tento signál je obecného typu \texttt{Data} reprezentující poslední informace odeslané OctoPrintem.

Pro tuto vrstvu jsem vytvořil další obalovou třídu nazvanou\\*\texttt{OctoPrintPushEvents}, která se stará o zpracování generických dat.
Jako veřejné rozhraní poskytuje konstruktor vyžadující \acrshort{url} tiskárny.
Druhým veřejným rozhraním je metoda \texttt{temperatures()}, která se stará o zpracování dat z OctoPrint.
Metoda vrací \texttt{Cold} signál s teplotami podložky a trysky.

Tato metoda ve své implementaci vytváří spojení pomocí třídy \texttt{WebSocket} a zpracová data z vráceného signálu.
Tyto data se nejdříve převedou na text, očistí se od informací o komunikaci a rozdělí se na řádky.
V každém řádku by se nyní měl vyskytovat platný \acrshort{json} objekt.

Tento objekt je následně zpracován na objekty \texttt{Bed} a \texttt{Tool}, které reprezentují teploty podložky a trysky.

\subsubsection*{Propojení s ViewModelem}

K implementaci ve \texttt{ViewModelu} stačí pouze vytvořit instanci\\*\texttt{OctoPrintPushEvents} s adresou tiskárny a zpracovat data signálu.

Zpracování dat se provádí stejně jako u běžných požadavků knihovny \texttt{Moya}, nebylo tedy potřeba přidávat žádnou další logiku.
