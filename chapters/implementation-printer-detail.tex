\section{Detail a přehled tiskárny}

V momentě, kdy má uživatel přidanou tiskárnu s platným přístupovým tokenem, může se jejím výběrem ze seznamu dostupných tiskáren dostat na detail.
Obrazovka detailu se skládá ze tří hlavních záložek, z nichž každá pak v rámci svých funkcionalit prezentuje další obrazovky.

První částí, která se uživateli zobrazí je přehled tiskárny.
Přehled je klíčovou funkcí aplikace.
Umožňuje totiž velmi rychle získat základní údaje o probíhajícím tisku.

\subsection{Připojení tiskárny}

OctoPrint se může nacházet ve stavu, kdy je připravený k použití, ale není připojen k tiskárně.
V tuto chvíli nelze získat informace o aktuálním tisku ale ani o tiskárně samotné.

Z tohoto důvodu jsem implementoval \uv{prázdnou obrazovku}, která uživatele upozorňuje na odpojenou tiskárnu a vyzývá ho k jejímu připojení.
Fakt, že data nejsou dostupná se propaguje z \texttt{ViewModelu}.
Ten vysílá signál nazvaný \textit{contentIsAvailable} booleanovského typu.
Obrazovka s výzvou k připojení tiskárny je dostupná právě když hodnota signálu je \textit{false}.

Pokud se uživatel rozhodne tiskárnu připojit, může k tomu využít tlačítko \uv{Připojit}.
To zařídí otevření nové obrazovky.
Na nové obrazovce může uživatel ze seznamu dostupných portů vybrat, k jaké tiskárně se má OctoPrint připojit.
Běžně tento seznam bude obsahovat právě jednu položku, protože každou tiskárnu obsluhuje právě jedna instance OctoPrint.

\subsection{Aktuální tisk}

V momentě kdy je tiskárna v pořádku připojena k OctoPrint rozhraní, zobrazuje aplikace informace o jejím stavu.
Součástí těchto informací jsou teploty podložky a trysky, aktuální tisk a video stream.

Protože všechny tyto informace nejsou dostupné z jediného koncového bodu, aplikace požadavky řetězí.
Postupně jsou volány požadavky pro aktuální stav tiskárny, aktuální tisk, informace o trysce a informace o podložce.
Implementace řetězení je vidět v ukázce \ref{code:detail-requests-chain}.

\swiftcode{code:detail-requests-chain}{Řetězení požadavků v Moya}{assets/code/detail-requests-chain.swift}

Řetězení pomocí reaktivního přístupu přináší možnost chyby řešit pomocí \texttt{railway} strategie.
Tedy řetězení probíhá po \uv{úspěšné koleji} až do výskytu první chyby.
Chyba slouží jako výhybka z této koleje a dále pokračuje po \uv{chybné koleji}.
Tento přístup umožňuje nezabývat se v jednotlivých požadavcích chybami.
Jakmile se v jednom požadavku vyskytne chyba, další požadavky se nevykonají.
Chybu tedy stačí ošetřit na jednom místě.
Více informací o této strategii je dostupné na \cite{fsharp-railway-strategy}.

Video stream aktuální tisku tisku je implementovaný pomocí pravidelného stahování obrázku tisknutí.
Aby aplikace nezatěžovala zbytečně síťové rozhraní, obrázek stahuje nejdříve pět vteřin po poslední aktualizaci.

\subsection{Využití síťových socketů}

\todo{Sekce o napojení detailu na sockety}
