\section{Lokální testování}

Protože aplikace komunikuje výhradně s tiskárnou musí být tiskárna dostupná i během vývoje.
Abych nemusel tiskárnu pořizovat, využil jsem možnost vytvořit virtuální tiskárnu.

S využitím technologie Docker jsem vytvořil kontejner s nainstalovanou aplikací OctoPrint.
Pomocí konfiguračního souboru jsem povolil vytvoření virtuální tiskárny.

OctoPrint byl takto spuštěný lokálně na mém počítači.
Po připojení OctoPrint k virtuálnímu portu tiskárny jsem mohl vyvíjet aplikaci i bez nutnosti vlastnit tiskárnu.

V ukázce \ref{code:docker} demonstruji spuštění kontejneru s virtuální tiskárnou z příkazové řádky svého počítače.

\begin{listing}[htbp]
\caption{\label{code:docker}Spuštění virtuální tiskárny pomocí Docker}
\begin{minted}[linenos, bgcolor=bgcode]{bash}
docker run -p"3200:5000" josefdolezal/virtuprint-docker 
\end{minted}
\end{listing}
