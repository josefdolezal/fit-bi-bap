\section{Nastavení}

Jako třetí obrazovku detailu tiskárny jsem zvolil nastavení.
Tato obrazovka obsahuje funkcionality, u kterých předpokládám, že budou využívány s menší intenzitou.
Funkce nastavení jsou rozděleny do více obrazovek, těm se samostatně věnuji v jednotlivých bodech níže.

Pod seznam funkcionalit jsem také umístil tlačítko pro zavření detailu tiskárny.
Po stisknutí tlačítka se obrazovka zavře a zobrazí se opět seznam tiskáren.

\subsection{Správa log souborů}

Obrazovku pro správu log souborů jsem implementoval opět pomocí tabulky.
Při příchodu na obrazovku stáhne ViewModel seznam logů a uloží je do lokální databáze.
Z důvodů urychlení stahování a minimalizace přenesených dat se nejprve stahuje pouze seznam souborů, obsah nikoli.

Pro stažení souboru musí uživatel otevřít novou obrazovku s detailem logu.
V tuto chvíli se spustí stahování logového souboru, který je po dokončení stažení zobrazen uživateli.

Z této obrazovky je také možné soubor smazat.

\subsection{Správa slicing profilů}

Správa slicing profilů je rozdělena do dvou částí.
Na první obrazovce se nachází seznam slicerů.
Z tohoto seznamu může uživatel výběrem sliceru přejít na seznam slicovacích profilů.
Seznam profilů odpovídá vybránému sliceru.

Dále je možné přejít na detail profilu, kde ho uživatel může upravit či smazat.
Aplikace také umožňuje vytvořit profil nový.
