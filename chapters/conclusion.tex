Cílem práce bylo vypracování mobilní aplikace pro platformu iOS umožňující nastavovat 3D tiskárny a tisknout z nich.

Výsledná aplikace umožňuje zobrazit přehled aktuálního tisku, správu souborů a nastavení 3D tiskárny.
Pomocí aplikace je také možné sledovat video stream z webové kamery připojené k tiskárně.
Uživatel také může vybrat soubor ve svém zařízení, nahrát ho do tiskárny a následně vytisknout.

V aplikaci lze komunikovat s libovolnou tiskárnou dostupnou na místní síti, která je aplikací automaticky nalezena.
Uživatel může velmi jednoduše takto tiskárnu přidat a obsloužit celý tisk.

\bigskip

Během implementace jsem se snažil většinu pozornosti věnovat architektuře aplikace.
Mnoho částí jsem proto v průběmu implementace několikrát přepsal.

Velkou výzvou bylo implementovat aplikaci reaktivně.
S reaktivním programování jsem doposud neměl žádné zkušenosti.
Nyní jsem ale rád, že jsem se takto rozhodl.
Využití reaktivního přístupu mi ušetřilo mnoho práce a velmi pravděpodobně i mnoho chyb, které bych jinak standardní cestou do aplikace zanesl.

\bigskip

Jsem přesvědčený, že díky využití zvolené architektury jsem také zvýšil testovatelnost aplikace.
Přesto, že jsem testy aplikace nikdy dříve nepsal, nebylo složité je s toutou architekturou vytvořit.

\subsection*{Výhled do budoucna}

V budoucnu je možné do vývoje aplikace přizvat programátory pohybující se pravidelně v prostředí 3D tisku a rozšířit ji tak v komunitě.
Z tohoto důvodu jsou zdrojové kódy práce volně dostupné jako open source.
Aplikaci je díky tomu možné rozšířit o nové funkcionality v případě aktualizace API OctoPrint.

V současné době ale neplánuji aplikaci distrubuovat běžným uživatelům obchodem App Store.
