\section{Ovládání tiskové hlavy}

Při implementaci ovládání jsem se inspiroval rozhraním OctoPrint, které je znázorněno na obrázku \ref{fig:octoprint-controls}.
Pohyb hlavy je možný po všech osách.
Aby bylo uživatelské rozhraní intuitivní, sjednotil jsem osy $x$ a $y$ do ovladače pro jednu ruku a osu $z$ do ovladače pro druhou.

\image{fig:octoprint-controls}{Ovládání tiskové hlavy v prostředí OctoPrint}{assets/implementation-octoprint-controls.png}

Rozhraní je tvořeno pomocí tlačítek, kde každá osa má jedno tlačítko pro pohyb v kladném a jedno pro pohyb v záporném směru.
Dále každý z ovladačů obsahuje tlačítko \uv{Domů}, které zavede hlavu 

\subsection{Pohyb hlavy}

Jelikož jsou prvky tvořeny tlačítky, je pohyb implementovaný pomocí požadavků pro každou interakci zvlášť.
\texttt{Controller} pro každý stisk vytvoří vstup \texttt{ViewModelu}, ten následně vytváří požadavky na tiskárnu.

Požadavek pro posun hlavy je implementovaný genericky parametrizovaný množinou os a směrem.
Aby UI nebylo zbytečně komplikované, rozhodl jsem uživateli neumožnit explicitně zvolit o jakou vzdálenost se má hlava posunout.
Každý požadavek tedy posune hlavu po dané ose o 10 mm vpřed nebo vzad.

Každý uživatelský vstup je namapovaný na právě jeden vstup \texttt{ViewModelu}.
Tím jsem dosáhl jednoduchého rozhraní \texttt{ViewModelu} a odstranil potřebu využití složitých řídících konstrukcí (switch konstrukce či řetězené podmínky).
Implementace je znázorněna v ukázce \ref{code:printer-controls-request}.

\swiftcode{code:printer-controls-request}{Implementace pohybu tiskové hlavy}{assets/code/printer-controls-request.swift}

\subsection{Ovládání trysky}

Z důvodu možnosti čištění trysky jsem přidal na tuto obrazovku tlačítko pro manuální tavení filamentu.
Obdobně jako u pohybu tiskové hlavy ani zde jsem neumožnil uživateli zvolit množství.
Při uživatelské interakci (stisku tlačítka \uv{Čistit}) aplikace vždy nechá roztavit 5 mm.

Implementace tohoto požadavku se velmi blízce podobá požadavkům na pohyb tiskové hlavy.
