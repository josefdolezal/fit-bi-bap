\section{Testy uživatelského rozhraní}\label{testovani-ui}

Testování uživatelského rozhraní si klade za cíl ověřit správné sestavení komponent grafického rozhraní.
Pomocí interakce s komponentami se také zkoumá, jakým způsobem komponenty reagují.
Na rozdíl od testů chování přistupují tyto testy k aplikaci jako k celku a zacházejí s ní obdobně jako by s ní zacházel uživatel. Tyto testy tedy nemají přístup k vnitřní implementaci aplikace.
Jelikož nevyžadují během chodu zásah člověka (test \textit{nahrazuje} jeho přítomnost), mohou být pouštěny automaticky.
Standardně se tedy pouští při implementaci každé nové funkce, mnohdy až několikrát denně. \cite{apple-ui-testing}

Protože tyto testy z jsou v mém případě pouze nadstavbou nad \textit{testy chování} vysvětlené níže, rozhodl jsem se je implementovat pomocí referenčních obrázků.
Testy tedy pro každý podstatný krok scénáře obsahují referenční obrázek, jak by obrazovka měla v danou chvíli vypadat.
Pokud se vzhled s referenčním obrázkem shoduje, test projde.
Nevýhodou tohoto přístupu je nutnost přegenerování referenčních obrázků v momentě, kdy se vzhled obrazovky (úmyslně) změní byť o jediný obrazový bod.
Podstatnou výhodou tohoto přístupu je ale nezávislost na implementaci.
Pokud se implementace změní, s velkou pravděpodobností to výsledky testů neovlivní.