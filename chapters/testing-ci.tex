\section{Průběžná integrace}

Průběžná integrace (z anglického \textit{Continous integration}, zkráceně \texttt{CI}) je praktika vývoje software, při které členové týmu integrují svou práci mnohdy až několikrát denně.
Každá integrace je automaticky ověřena kompilací na buildovacím serveru a spuštěním testů.
Tato technika si klade za cíl odhalid integrační chyby aplikace co nejdříve je to možné.
To má za následek snížení časové náročnosti implemetace nových funkcí bez rizika narušení funkcionalit původní aplikace.

Nové funckionality se běžně skládají z nového kódu, upraveného produkčního kódu a upravených testů.
V momentě kdy vývojář označí funkcionalitu za kompletní, odešle své změny na vzdálený buildovací server, kde se aplikace zkompiluje a otestuje.
Integrace je následně provedena jen v případě, že celý proces proběhl bez chyby.
Jak je zjevné, při této technice je zásadní vysoké pokrytí aplikace testy.

Průběžné integrování je často velmi blízce vázáno na správu zdrojových kódů.
Některé systémy jako GitHub či GitLab je dokonce možné nastavit tak, aby by změny byly přidány do hlavní větve až ve chvíli, kdy build projde bez komplikací. \cite{travis-ci-building-pr}

Pro průběžnou integraci je na trhu dostupných mnoho řešení.
Rozhodoval jsem se mezi použitím komerčních řešení \textit{Travis CI} a \textit{Circle CI}.
První zmíněná možnost je v open-source komunitě velmi populární a dalo by se říct, že je pro \texttt{CI} téměř synonymem.
Podle nezávislého průzkumu z roku 2016 využívá \textit{Travis CI} téměř pětinásobek uživatelů než ostatních poskytovatelů dohromady. \cite{oregonstate-ci-survey}

Druhé řešení jsem chtěl vyzkoušet kvůli rostoucí popularitě. \cite{circleci-popularity}
Bohužel je kompilace na operačním systému macOS pro open-source projekty možná až po individuálním schválení. \cite{circleci-pricing}
Z tohoto důvodu jsem nakonec zvolil první řešení, kde je kompilace na systému macOS pro studenty zdarma.

\subsection{Statická analýza kódu}

Spolu s roustoucím týmem a rostoucím zdrojovým kódem se často zavádí směrnice programování.
Tyto směrnice udávají jakým způsobem je kód formátovaný.
Při dodržování těchto směrnic se kód stává konzistentním a dobře čitelným, to má za následek zrychlení vývoje a zvyšuje udržitelnost kódu.
Aby nebylo nutné analyzovat kód ručně, existují nástroje které analýzu automatizují.

Pro zajištění konzistence jsem použil nástroj \texttt{SwiftLint} od společnosti \texttt{Provider}.
\texttt{SwiftLint} je nástroj obsahující sadu pravidel pro statickou analýzu kódu.
Mimo jiné kontroluje správné zarovnání kódu, délky řádků či názvy proměnných.
V době psaní práce obsahuje tento nástroj dohromady téměř osmdesát pravidel.
Ve své implemetaci jsem se rozhodl vynechat dvě pravidla, které jsem nepovažoval za stěžejní a jejich dodržování by z mého pohledu vedlo ke snížení čitelnosti.
Kromě těchto dvou pravidel jsem využil i možnost vypnout pravidla pro určité části kódu.
Nejčastěji se jednalo o pravidla na kontrolu délky metod.

Tuto analýzu zajišťuje buildovací server před začátkem kompilace.
Pokud nejsou v kódu závažné porušení směrnice, pokračuje se ke kompilaci.
V opačném případě nastane chyba a je nutné kód opravit a integraci pustit znovu.
