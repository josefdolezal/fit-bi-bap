\section{Testy chování}\label{testovani-bdd}

Chování modolu aplikace lze otestovat tak, že se zkoumá plnění jeho závazků.
Testované objekty mají definované rozhraní a závisloti, tím mají také deklarovány i závazky.
Závazky určují, jakým způsobem by měl objekt působit na zbylé části aplikace a jaké schopnosti a funkce má.
Ze schopností a funkcí lze odvodit jakým způsobem se má objekt chovat.

\subsection{Testování komponent}

Pomocí těchto testů lze ověřovat funkčnost libovolné komponenty aplikace.
Složitost a rozsah těchto testů se následně odvíjí od počtu závazků komponenty.
Testování probíhá tak, že vlastnosti objektu jsou pomocí rozhraní měněny.
Přitom se sleduje, jestli se objekt na základě změn chová podle očekávání. \cite{objcio-bdd}

\subsection{DSL}

Testy popisuje \acrfull{dsl}.
To je jazyk který pomocí kombinace klíčových slov a textového popisu definuje jak se komponenta má v určitou chvíli chovat.
Více o \acrshort{dsl} lze zjistit na \cite{petrikainulainen-dsl}.

V běžném testovacím rozhraní prostředí Xcode tento jazyk nenabízí.
Rozhodl jsem pro to využí dvou knihoven.
Pomocí Knihovny \textit{Quick} jsem mohl \acrshort{dsl} v projektu využít.
Očekávané hodnoty jsem ověřoval pomocí \textit{Nimble}.

\subsection{Testování ViewModelu}

Tento princip využívám ve své implementaci pro testování Modelu a ViewModelu.
Díky dostatečnému rozsahu testů logiky aplikace lze usuzovat, že v produkčním nasazení se vyskytne jen nepatrné množství chyb.
\texttt{View} vrstvu následně není samostatně nutné testovat, protože z testů ViewModelu je téměř jisté, že data jsou správně připravena k zobrazení.

U ViewModelu jsem podle zvoleného scénáře navrhl \acrshort{dsl}.
Tímto jazykem jsem definoval jak očekávám, že s objektem bude pracovat.
Pomocí Nimble knihovny jsem dále ověřoval, jestli na vytvořené vstupy vytváří ViewModel očekávané výstupy jak je vidět v ukázce \ref{code:bdd-dsl}.

Během implementace jsem vytvořil bez mála sto testů, díky kterým jsem odchytil velké množství chyb.
Tyto chyby jsem často do aplikace zanesl v momentě, kdy jsem k již hotové obrazovce implementoval novou funkcionalitu.

\swiftcode{code:bdd-dsl}{Použití \acrshort{dsl} pro definici testu}{assets/code/bdd-dsl.swift}

\subsection{Vývoj řízený testováním chování}

S pojmem testování chování je velmi blízce spjat \textit{vývoj řízený testováním chování} (z anglického Behavior Driven Development, zkráceně \acrshort{bdd}).
V tomto přístupu k vývoji se před konkrétní implementací nejdříve nadefinuje, jakým způsobem se mají testované komponenty chovat.
Následně se sestavují testy, které vyžadují implementování vyžadovaného chování.
Tím má objekt definované, jaké rozhraní musí implementovat a jaké závislosti bude vyžadovat.

Průběh vývoje probíhá cyklicky.
Nejdříve se nadefinují testy některé funkcionality, následně se implementuje jen tolik kódu, kolik je nutné aby testy byly úspěšné.
Tento postup se opakuje dokud nejsou implementovány veškeré funkce vyžadované po objektu.
Více informací o tomto přístupu je dostupných na \cite{objcio-bdd}.

Tento přístup jsem využil na část ViewModelů.
Vzhledem k rozsahu implementace jsem ale většinu testů napsal až po kompletní implementaci ViewModelu.
