\section{Testy chování}\label{testovani-bdd}

Chování jednotlivých modulů aplikace se testuje pomocí zkoumání plnění svých závazků.
Testované objekty mají definované své rozhraní a závisloti, tím jsou deklarovány i závazky objektu, které musí splnit.
Závazky určují, jakým způsobem by měl objekt působit na zbylé části aplikace a jaké schopnosti a funkce má.
Ze schopností a funkcí lze odvodit jakým způsobem se má objekt chovat.
Testovaným objektem může být libovolný modul aplikace.
Složitost a rozsah testů se odvíjí od počtu závazků objektu.
Testování probíhá tak, že vlastnosti objektu jsou pomocí rozhraní měněny a sleduje se, jestli se objekt na základě změn chová podle očekávání. \cite{objcio-bdd}

V reálném světě si lze jako objekt představit auto.
Jako změnu vlastnosti můžeme lze použít vyčerpání nádrže.
Očekávaným chováním auta následně je, že přestane být pojízdné a rozsvítí kontrolku řidiči.
Auto projde testem chování, právě když při prázdné nádrži je nepojízdné a svítí kontrolka.

Tento princip testování využívám pro testování View Modelu a Modelu (viz. kapitola MVVM: Model-View-ViewModel).
Díky otestování dostatečného rozsahu logiky aplikace, lze usuzovat, že v produkčním nasazení se vyskytne jen nepatrné množství chyb.
View vrstvu následně není samostatně nutné testovat, protože z testů View Modelu je jisté, že data jsou správně připravena k zobrazení.
Testy pomocí porovnání skutečného vzhledu s očekávaným (zmíněno v testech uživatelského rozhraní) jsou tedy dostačující.

Obdobně jako testy rozhraní i tyto testy se pouštějí při vývoji až několikrát denně.
Slouží také jako dobrý ukazatel kvality aplikace a dokáží určit její rozsah.

S pojmem \textit{testování chování} je velmi blízce spjat \textit{vývoj řízený testováním chování} (z anglického Behavior Driven Development, zkráceně BDD).
V tomto přístupu k vývoji se před konkrétní implementací nejdříve nadefinuje, jakým způsobem se mají testované komponenty chovat.
Následně se sestavují testy, které vyžadují úplné implementování vyžadovaného chování.
Tím má objekt definované, jaké rozhraní musí implementovat a jaké závislosti bude vyžadovat.
Testy jsou popisovány takzvaným DSL.
To je jazyk který pomocí kombinace klíčových slov a textového popisu definuje jak se komponenta má v určitou chvíli chovat.
Ukázka \ref{code:bdd-dsl} znázorňuje DSL v testovacím frameworku Quick.
Jakmile jsou vytvořené testy, přechází se k implementaci.
Při implementování se vybere test, kterým má aplikace projít a implementuje se pouze takový kód, který zaručí průchod tímto testem.
Takto se postupuje, dokud objekt neprojde všemi testy. \cite{objcio-bdd}

\swiftcode{code:bdd-dsl}{Použití DSL pro definici testu}{assets/code/bdd-dsl.swift}
