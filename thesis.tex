% options:
% thesis=B bachelor's thesis
% thesis=M master's thesis
% czech thesis in Czech language
% slovak thesis in Slovak language
% english thesis in English language
% hidelinks remove colour boxes around hyperlinks

\documentclass[thesis=B,czech]{FITthesis}[2012/06/26]

\usepackage[utf8]{inputenc} % LaTeX source encoded as UTF-8
\usepackage{pdfpages}
\usepackage{graphicx} %graphics files inclusion
% \usepackage{amsmath} %advanced maths
% \usepackage{amssymb} %additional math symbols

\usepackage{dirtree} %directory tree visualisation

% % list of acronyms
% \usepackage[acronym,nonumberlist,toc,numberedsection=autolabel]{glossaries}
% \iflanguage{czech}{\renewcommand*{\acronymname}{Seznam pou{\v z}it{\' y}ch zkratek}}{}
% \makeglossaries

\newcommand{\tg}{\mathop{\mathrm{tg}}} %cesky tangens
\newcommand{\cotg}{\mathop{\mathrm{cotg}}} %cesky cotangens

% % % % % % % % % % % % % % % % % % % % % % % % % % % % % %
% ODTUD DAL VSE ZMENTE
% % % % % % % % % % % % % % % % % % % % % % % % % % % % % %

\department{Katedra informačních technologií}
\title{iOS aplikace k ovládání 3D tiskáren}
\authorGN{Josef} %(křestní) jméno (jména) autora
\authorFN{Doležal} %příjmení autora
\authorWithDegrees{Josef Doležal} %jméno autora včetně současných akademických titulů
\supervisor{Ing. Miroslav Hrončok}
\acknowledgements{V této části bych rád poděkoval každému, kdo mi během této práce pomohl.
Nejprve Miroslavovi Hrončokovi, vedoucímu mé práce, za podnětné rady, pomoc s korekturou a uvedení do problematiky 3D tisku.

Děkuji Tomášovi Sýkorovi za cenné rady v implementační části, za pomoc s automatizací testování a kompilace a za propůjčení vývojářských certifikátů pro podepisování aplikace.

Děkuji také své rodině za podporu a pevné nervy.
}
\abstractCS{Bakalářská práce se zabývá vytvořením mobilní aplikace pro ovládání 3D tiskárny.
Cílem projektu je naprogramování aplikace, která uživatelům zjednoduší ovládání 3D tiskárny z mobilních zařízení s operačním systémem iOS.
Důležitou součástí řešení je možnost přidání síťové tiskárny s minimální konfigurací.
Z důvodu možnosti rozšíření funkcionalit v budoucnu jsou zdrojové kódy volně dostupné.
}
\abstractEN{Bachelor thesis aims to create an application to control 3D printers.
The application simplifies control of 3D printers from devices running iOS operating system.
An important part of the solution is the ability add new printer with a minimum configuration steps.
Application is available as open source to allow community to possibly add new functions in the future.
}
\placeForDeclarationOfAuthenticity{V~Praze}
\declarationOfAuthenticityOption{4} %volba Prohlášení (číslo 1-6)
\keywordsCS{mobilní aplikace, 3d tisk, octoprint, iOS, Swift, open source
}
\keywordsEN{mobilní aplikace, 3d tisk, octoprint, iOS, Swift, open source
}
\website{https://github.com/josefdolezal/fit-bi-bap}

\begin{document}

% \newacronym{CVUT}{{\v C}VUT}{{\v C}esk{\' e} vysok{\' e} u{\v c}en{\' i} technick{\' e} v Praze}
% \newacronym{FIT}{FIT}{Fakulta informa{\v c}n{\' i}ch technologi{\' i}}

\begin{introduction}
  V posledních letech můžeme být svědky velkého technologického rozmachu.
Nedílnou součástí tohoto rozmachu jsou mobilní aplikace a také 3D tiskárny.
V oblasti 3D tiskáren se popularita zvýšila za posledních pět let více než osmkrát (měřeno dle zájmu o téma v Google vyhledávání) \cite{3d-print-google-trends}.
Mnohem většího úspěchu dosáhly mobilní zařízení, ktrá v září 2016 poprvé přesáhli využívání osobních počítačů \cite{mobile-devices-market-share}.

I navzdory těmto faktům je v současné době využívání a ovládání 3D tiskáren doménou zejména počítačů.
O tom svědčí především rozšíření mobilních aplikací.
Ty jsou v současné době pouze dvě, jejich funkcionalita je navíc oproti desktopové verzi značně omezená.

Ve své bakalářské práci se zabývám návrhem mobilní aplikace pro ovládání 3D tiskáren prostřednictví rozhraní OctoPrint, její analýzou a implementací.
Práci uvádím kapitolou *3D tisk*, která čtenáře stručně uvede do problému.
V kapitole *Analýza* zkoumám možné technologie a paradigmata, z nichž vybrané zmiňuji následně v kapitole o implementaci.
V části věnující se návrhu popisuji jaké funkcionality implementovaná aplikace bude obsahovat a způsob jakým jsem je do aplikace zakomponoval.
O udržení kvality aplikace a využití automatizace v průběhu vývoje mluvím v kapitole *Testování*.

Výsledná aplikace je vedena jakou open source.
Její zdrojový kód je tedy volně dostupný s možností libovolných úprav.
Přestože aplikace není distribuována standardní cestou obchodem App Store, je na tuto formu distribuce plně připravena.
Jednotilivé verze aplikace jsou v tuto chvíli distribuovány pouze registrovaným vývojářům pomocí Fabric Beta.

\end{introduction}

\chapter{3D tisk}\label{3d-tisk}

V této kapitole stručně zmiňuji princip a technologii 3D tisku.
Čtenáře uvedu také do stávajícího stavu z pohledu ovládání 3D tiskáren a využití mobilních aplikací k tomuto účelu.

\section{Technologie 3D tisku}\label{3d-tisk-technologie}

3D tisk je proces, při kterém lze z digitálního modelu vytvořit reálný objekt.
Tyto objekty vznikají postup nanášením jednotlivých vrstev modelu na sebe ve vertikálním směru.
Jedna z metod využívajících vrstvení materiálu se nazývá Fused deposition modeling (zkráceně FDM).
Princip FDM spočívá v tavení plastu (případně jiných materiálů) pomocí tiskové hlavy, která následně nanáší taveninu ve vrstvách na sebe \cite{3d-print-fdm}.
Takto vytvořené plastové objekty mají velkou výhodu v nízkých nákladech na výrobu.
Jsou tedy ideálním prostředkem k výrobě prototypů nebo produkování omezeného množství výrobků \cite{3d-print-for-prototyping}.

\section{OctoPrint - Ovládání 3D tiskárny}\label{3d-tisk-ovladani}

Tisk lze obsloužit pomocí mnoha aplikačních rozhraní.
Velmi oblíbeným nástrojem je OctoPrint, který nabízí ovládání pomocí webového prohlížeče.
Jedná se tedy o webové rozhraní, které reprezentuje uživateli příkazy tiskárny pomocí tlačítek či jiných grafických prvků.
Z prohlížeče je tak možné měnit základní vlastnosti jako např. teploty, průběh tisku či nastavení připojení k tiskárně.
Bohužel je ale toto rozhraní zcela zaměřeno na použití z počítače.
Ovládací prvky nejsou dostatečně veliké a manipulace s nimi přináší velmi špatnou uživatelskou zkušenost na mobilních zařízeních.
Možné alternativy v kapitole \textit{Analýza}.

Kromě webového rozhraní nabízí OctoPrint také API pro aplikace třetích stran.
Pro implementaci své aplikace využiji právě tohoto rozhraní.

V době psaní práce je dostupná verze API $1.3$, která oproti předchozí verzi nabízí nové funkce jako např. správu uživatelů či nastavení systému.
Ty ale sám autor označuje za experimentální.
Pro mobilní aplikaci navíc nejsou stěžejní.
Z tohoto důvodu jsem se rozhodl využít starší verzi $1.2.15$, která většinu funkcí označuje jako stabilní.

\section{Definice pojmů}\label{3d-tisk-definice-pojmu}

Přestože se obvykle slovem tisk rozumí běžný papírový tisk pomocí inkoustové nebo laserové tiskárny,
ve své práci tento druh tisku vůbec neuvažuji a pojmem tisk rozumím 3D tisk.
Analogické předefinování si dovolím i u pojmů odvozených.

Ve své práci uvažuji jako metodu tisku pouze FDM zmíněnou v kapitole \textit{Technologie 3D tisku}.


\chapter{Analýza a rešerše}\label{analyza}

\section{Architektura apliakce}\label{analyza-architektura}

Jako doporučenou architekturu aplikací pro platformu iOS (konkrétně iPhone a iPad) uvádí Apple Model-View-Controller (zkráceně MVC).
Přestože je MVC pro vývoj aplikací nejpopulárnější, rozhodl jsem před začátkem implementace prozkoumat i jiné existující architektury.
Z alternativních architektur jsem nakonec zvolil Model-View-ViweModel, kterou porovnám s dopručeným MVC.
V závislosti na výsledku porovnání zvolím ideální architekturu pro svou aplikaci.

\subsection{MVC: Model-View-Controller}\label{analyza-mvc}
Tato architektura rozděluje aplikaci do tří vrstev: Model, View a Controller.

\subsubsection{Popis architektury}

\begin{description}
  \item[Model] reprezentuje perzistentní objekty, které aplikace využívá pro vnitřní logiku a prezentaci dat uživateli.
  Každý modelový objekt může být v relaci s libovolným počtem jiných modelových objektů.
  Tato vrstva je často reprezentována databází, příkladem mohou být databáze CoreData, Realm nebo SQLite.

  \item[View] je datový objekt viditelný uživatelem. View obsahuje logiku pro vykreslení a interakci s uživatelem.
  Přestože se View standardně používá pro zobrazení modelových objektů nebo jejich úpravu, jsou od sebe tyto vrstvy striktně odstíněny.
  Na platformě iOS tuto vrstvu reprezentuje framework UIKit vytvořený Applem.

  \item[Controller] je aplikační vrstva, která na základě vstupů z View aktualizuje a mění Model nebo překresluje View v případě, že zobrazovaná data už nejsou aktuální.
  Jedním z úkolů Controlleru je striktně zamezit přímé interakci mezi View a Modelem.
  Toto oddělení je zavedeno proto, aby View nemuselo znát konkrétní strukturu Modelu a aby Model nemusel obsahovat logiku formátování dat (cena, čas, ...) pro vykreslení.
  Dále se stará o navigaci mezi obrazovkami, síťování a interakci s uživatelem.
  Při rozdělení do obrazovek platí pravidlo, že jeden controller obsluhuje jedno nebo více View.
  Ke korektnímu vykreslení View využívá libovoné množství modelových objektů.
  O jednu obrazovku se typicky stará právě jeden Controller, je ale možné jich použít více.
\end{description}

\begin{figure}\centering
	\includegraphics[width=0.5\textwidth]{assets/mvc-architecture.png}
	\caption[Architektura MVC]{Architektura MVC}\label{fig:architektura-mvc}
\end{figure}

Z tohoto shrnutí vyplývá, že Controller je velmi blízce spjat s View. Toto propojení reprezentuje obrázek \ref{fig:massive-mvc}.

\begin{figure}\centering
	\includegraphics[width=0.5\textwidth]{assets/mvc-massive-view-controller.png}
	\caption[Role Controlleru v MVC]{Controller spjatý s View}\label{fig:massive-mvc}
\end{figure}

\subsubsection{Modelový příklad použití}

Pro možné porovnání architektury jsem připravil scénář stažení libovolných dat na základě požadavku uživatele.
V MVC by se architektura chovala takto:

\begin{itemize}
  \item Uživatel v aplikaci klikne na tlačítko \uv{Stáhnout data}.
  \item Tuto interakci odchytí view a upozorní controller.
  \item Controller na základě upozornění stáhne data a předá je modelu k uložení.
  \item Model ukládá data a notifikuje controller o změně.
  \item Controller aktualizuje view.
  \item Nastane-li během stahování chyba, controller vytváří nové view a chybu prezentuje uživateli.
\end{itemize}

\subsubsection{Shrnutí}

Shrneme-li vlastnosti vrstev, jejich klíčové role jsou:

\begin{itemize}
  \item Model udává, jakým způsobem jsou data uložena,
  \item View se stará o správné vykreslení předformátovaných dat,
  \item Controller se stará o ostatní logiku.
\end{itemize}

Pro zmíněné notifikace nabízí Apple řešení pomocí Delegate pattern.
Controller musí naimplementovat specifické rozhraní, čímž se stane delegátem.
Jako delegát se pak může zaregistrovat na notifikace obejktů, jejichž rozhraní implementoval.

MVC je v době psaní této práce nejpoužívanější architekturou a to především díky své jednoduchosti.
Při tvorbě větších aplikací ale nemusí být vhodné.
Controller se při nestandradním grafickém návrhu může stát velmi složitým, což výrazně snižuje jeho čitelnost a testovatelnost.
Z tohoto důvodu se MVC často přezdívá \uv{Massive View Controller}.
Díky přímému napojení controlleru na View se při testování chování Controlleru (behavioral testing) musí využít simulátoru mobilního operačního systému a aplikaci v něm automaticky \uv{proklikat}.
To zvyšuje časovou náročnost testování, dokonce v některých případech znemožňuje testování úplně (controlleru nezle podvrhnout mock objekty).
Tento problém se snaží řešit architektura MVVM od společnosti Microsoft.

\subsection{MVVM: Model-View-ViewModel}\label{analyza-mvvm}

Z důvodu nárustu nároků na mobilní aplikace se v posledních letech rozmáhá architektura MVVM.
Tato architektura vychází ze zmíněného MVC a jejím základním úkolem je zjednodušit Controller.

\subsubsection{Popis architektury} \label{architektura-mvvm-popis}

Za účelem zjednodušení Controlleru se ke stávajícím třem vrstvám přidává View Model, který se stará o přípravu dat z Modelu pro zobrazení a také o perzistenci změn.

\begin{description}
  \item[View Model] je objekt vlastněný controllerem za pomoci kompozice.
  Pro Controller připravuje naformátované výstupy a poskytuje mu rozhraní pro vstupy.
  Výstupem se rozumí veškerá data, která jsou potřebná pro sestavení View.
  To může být např. datum ve specifickém formátu, cena včetně měny nebo informace o tom, kolik řádků bude obsahovat tabulka na obrazovce.
  Oproti MVC tedy perzistentní data nejsou viditelná Controlleru, ale pouze View Modelu.
  Ten je nejdříve připraví pro zobrazení.
  Vstupem může být libovolná interakce uživatele:
  změna textu v textovém poli, stisknutí tlačítka, ale i fyzický pohyb telefonem (otočení obrazovky).
  Na základě vstupů spouští View Model svou vnitřní logiku a generuje výstupy.
\end{description}

Zodpovědnost Controlleru se zavedením View Modelu dramaticky snižuje.
V ideálním případě je Controller zodpovědný pouze za správné sestavení View a napojení zfromátovaných výstupů na něj.
Dále pak za odchycení uživatelských interakcí a jejich propagaci do View Modelu.
Toto chování zachycuje obrázek \ref{architektura-mvvm}.

\begin{figure}\centering
	\includegraphics[width=0.5\textwidth]{assets/mvvm-architecture.png}
	\caption[Architektura MVVM]{Architektura MVVM}\label{architektura-mvvm}
\end{figure}

\subsubsection{Modelový příklad použití} \label{architektura-mvvm-priklad}

Pro porovnání architektury s MVC lze opět využít scénář pro stažení dat. Pro tento scénář by se architektura MVVM chovala následovně:
\begin{itemize}
  \item Controller napojuje výstupy view modelu na view a vytváří pravidla pro převod uživatelské interakce na vstupy view modelu.
  \item Uživatel v aplikaci klikne na tlačítko "Stáhnout data".
  \item View upozorňuje controller na interakci uživatele, ten automaticky vytváří vstup pro view model.
  \item View model na základě vstupu stahuje data a předává je modelu.
  \item Model po uložení notifikuje view model, ten vytváří výstup pro controller, který nechává překreslit view.
  \item V případě chyby vytvoří view model chybový výstup, ten se pomocí controlleru propaguje do view.
\end{itemize}

\subsubsection{Shrnutí} \label{architektura-mvvm-shrnuti}

Vrstvy mají následující klíčové vlastnosti:
\begin{itemize}
  \item Model definuje jakým způsobem jsou data uložena a při změně notifikuje View Model,
  \item View vykresluje na obrazovku naformátované výstupy a upozorňuje controller při interakci uživatele,
  \item Controller sestavuje hierarchii view, napojuje zformátované výstupy view modelu na view a z uživatelské interakce vytváří vstupy pro view model,
  \item View model načítá data modelu, na základě vstupů z controlleru nebo změny modelu generuje výstupy pro controller.
\end{itemize}

Oproti MVC je na tomto příkladu vidět snížení zodpovědnosti controlleru. Tato zodpovědnost se přesunula do View Modelu.
Na první pohled nemusí být tato změna opodstatněná, protože logika aplikace nezmizela, jen se přesunula.
Právě to ale umožnilo (nebo minimálně zjednodušilo) způsob, jakým lze logiku testovat.
View Model generuje výstupy na základě vstupů, v testech tedy lze uživatelskou interakci podvrhnout a testovat pouze výstupy (není potřeba vytvářet View ani Controller).
Dodatečně lze otestovat i uživatelské rozhraní.
Protože logika aplikace je otestována pomocí testů View Modelu, uživatelské rozhraní už stačí otestovat např. shodou View s referenčním obrázkem.

Při pohledu na notifikace je vidět, že přibyl typ, který nebyl v MVC potřeba.
Jedná se o notifikace směrem z View Modelu ke Controlleru (View Model nemá referenci na Controller, nemůže ho notifikovat přímo).
Některé výstupy view modelu je tedy potřeba sledovat v čase a na jejich změny reagovat.
Toto lze vyřešit pomocí KVO, které nabízí Apple v základu.
KVO umožňuje objektu zaregistrovat se na notifikace o změně stavu nějakého libovolného objektu.
V případě Controlleru by se registroval na změny stavu výstupů View Modelu.
Kdykoliv by se výstup změnil, Controller by dostal notifikaci.
Tento přístup ale není běžný pro použití s jazykem Swift.
Tento postup navíc neřeší synchronizaci vláken, z tohoto důvodu by mohlo docházet k nekonzistenci dat či neočekávanému chování.
Místo KVO se nyní standardně používají reaktivní rozšíření, které popisuji v následujících kapitolách.

Přestože mnou implementovaná aplikace není v ohledu na uživatelské scénáře nijak složitá, obsahuje mnoho obrazovek.
Obrazovky jsou vysoce interaktivní a více se k jejich implementaci hodí reaktivní přístup.
Z tohoto důvodu jsem jako architekturu vybral MVVM s použitím reaktivních rozšíření místo standardního MVC.

\section{Synchronizace vláken aplikace} \label{analyza-synchronizace-vlaken}

Mobilní aplikace se svou povahou značně liší od běžných webových nebo desktopových aplikací.
Oproti zmíněným jsou často mnohem interaktivnější, tedy ovládané nejen běžnými vstupy, ale i vlastnostmi zařízení (GPS poloha, orientace zařízení, dostupnost periferií).
Aby tyto vstupy nekazily uživatelský zážitek, jsou v systému implementovány asynchronně.

Jako příklad lze vzít získávání polohy z družic.
To provádí telefon na vlákně v pozadí.
Tím je zaručené, že hlavní vlákno aplikace (které se mimo jiné stará o vykreslování) není blokované a s aplikací lze bez problému intereagovat.
Jakmile jsou dostupné informace o poloze, aplikace zažádá systém o prostředky na hlavním vlákně a získanou polohu prezentuje uživateli.

Tento přístup je v SDK poskytovaným Applem velmi obvyklý a využívá ho mnoho standardních knihoven.
Ne vždy je ale běžné, aby výsledek byl prezentovaný na hlavním vlákně.
Takový princip je využit u síťových požadavků.

Síťový požadavek se vykonává na vlákně v pozadí a na stejném vlákně je zpracována i odpověď ze vzdáleného serveru.
Je-li potřeba odpověď zpracovat na hlavním vlákně, musí vývojář explicitně čas na hlavním vlákně vyžádat pomocí Grand Central Dispatch či vlákna synchronizovat jiným spůsobem.
Od verze systému iOS 9 je navíc nemožné udělat síťový požadavek synchronně (blokovat vlákno, které požadavek vytvořilo až do konce zpracování) \cite{apple-whats-new-in-ios}.

Otázkou tedy zůstává jak správně synchronizovat všechny vlákna, která potřebují zpracovat data na hlavním vlákně.

Při analýze tohoto problému jsem se zaměřil na tři možná řešení.
První dvě jsou implementovány přímo v dodávaném SDK, třetí možnost je knihovna od společnosti GitHub.

\subsection{Grand Central Dispatch}

Grand Central Dispatch (zkráceně GCD) je technologie vyvinutá společností Apple přinášející optimalizovanou podporu pro aplikace fungující na vícejádrových procesorech.
GCD je implementována nad standardními systémovými vlákny, vývojáři ale nabízí mnohem jednoduší rozhranní.

Pro zjednodušení je využit princip front (z angl. queue), které jsou reprenzentovány třídou \textit{DispatchQueue}.
DispatchQueue je implementována jako \textit{thread-safe}, je tedy možné k nim přistupovat ve stejný okamžik z několik vláken najednou \cite{apple-concurrency-programming-guide}.
Do této fronty je možné vkládat jednotky práce, GCD se postará o to aby byly spuštěny na ve správném pořadí.
V závislosti na konfiguraci je pak umožňuje jednotlivé úkoly spouštět synchronně nebo asynchronně.

V základu je dostupná \textit{main} fronta, která je synchronní a umožňuje vykonat práci na hlavním vlákně.
Pro synchronizaci vláken stačí požadované jednotky práce přidávat do správných front  \ref{code:create-dispatch-queue}.

Je-li potřeba vytvořit vlastní frontu (z důvodu uvolnění času na hlavním vlákně), stačí specifikovat název fronty a její prioritu.
Vytvoření nové fronty běžící v pozadí je vidět v ukázce \ref{code:create-dispatch-queue}.

\swiftcode{code:create-dispatch-queue}{GCD: Vytvoření vlastní fronty}{assets/code/create-dispatch-queue.swift}

Přidání do libovolné fronty..

\subsubsection{Quality of Service}

Pro možnost rozlišení priorit front využívá Apple Quality of Service (zkráceně QoS).
Díky QoS je možné určit, jakou prioritu bude mít danná fronta při rozdělování vláken.
V současné době existují čtyři QoS priority \ref{apple-prioritize-work-with-qos}:

\begin{description}
  \item[User-interactive] Práce, se kterou uživatel přímo interaguje.
  Tato fronta má nejvyšší prioritu, nejčastěji se v ní provádí překreslování uživatelského rozhranní či animace.
  Jednotlivé úkoly z fronty jsou vykonané ihned.
  \item[User-initiated] Práce, kterou zadal uživatel a vyžaduje okamžitý výsledek.
  Nejčastěji se stará o akce, které nastanou po interakci s některým z ovládacích prvků (např. tlačítko).
  \item[Utility] Práce, která vyžaduje více času pro své dokončení.
  Typicky se jedná o síťové požadavky nebo načítání dat.
  \item[Background] Práce s nízkou prioritou, kterou si nevyžádal uživatel.
  Využívá se zejména pro dávkové mazání souborů, synchronizaci nebo indexování databáze.
\end{description}

Při vytváření front je důležité správně volit prioritu.
Budou-li všechny fronty využívat jednu prioritu, může se stát, že systémová vlákna nebudou správně využita.
To by vedlo ke snížení výkonu aplikace a špatné uživatelské zkušenosti.

Přestože GCD je velmi zajímavá abstrakce nad standardními vlákny, jedná se stále o nízkoúrovňové API.
Mezi jednotkami práce nelze vyvářet závislosti či je řetězit.
Tuto možnost ale nábízí \textit{Foundation} framework pomocí \textit{NSOperationQueue}.

\subsection{NSOperationQueue a NSOperation}

NSOperationQueue a NSOperation je abstrakce nad metodou GCD zmíněnou víše.
Toto API je od verze systému iOS 4 implementováno právě pomocí GCD, nabízí ale nové možnosti přístupu k prováděným jednotkám práce.
Hlavním rozdílem oproti GCD je vyloučení pojmu \textit{vlákno}.
Jednotky práce nejsou nadále reprezentovány pomocí \textit{first class funkcí}, ale nově pomocí instancí tříd \textit{NSOperation} o jejichž správu se stará \textit{NSOperationQueue}. \cite{cocoacasts-nsoperation-vs-gcd}

\subsubsection{NSOperation}

Pro reprezentaci jednotek práce se standardní funkce v GCD nejevily jako dostatečné.
S představením NSOperationQueue, je ale nově možné nad opakovanými jednotkami abstrakci pomocí tříd.
Toho lze docílit pomocí vytváření vlastních tříd dědících od NSOperation.

NSOperation představuje samostatnou jednotku práce (operaci).
Jedná se o abstraktní třídu zajišťující \textit{thread-safe} přístup ke správě stavu, priority a závislostí.

Jako úkol lze chápat například jednotlivé síťové požadavdy, zpracování vstupů a uživatele a jejich perzistence nebo komplexní výpočty.
Úkolem je možné označit ale i libovolnou strukturovanou jednotku práce, u které je potřeba udržovat stav nebo zpracovat její datový výstup.

Oproti CGD má objektový přístup výhodu nejen v přehlednější strukturování, ale i v možnosti vytváření závislostí a správě stavu \cite{apple-operation-queues}.

\begin{description}
  \item[Závislosti] definují, jaké další operace musí být vykonány před tím, než bude daná operace spuštěna.
  Přidat lze libovolný počet dalších operací, tím lze dosáhnout komplexního procesu za pomoci skládání menších bloků.
  Jako příklad lze uvést síťovou komunikaci, kde každý požadavek závisí na ověření dostupnosti internetu (první operace) a na patřičném ověření uživatele (druhá operace).
  Ke každému požadavku následně stačí přiřadit tyto dvě operace jako závislost, tím je zaručené že požadavek se pošle jen když je dostupné připojení k internetu a zároveň je uživatel ověřený.
  \item[Správa stavu] umožňuje operaci pozastavit, spustit znovu nebo zrušit.
  Oproti GCD, kde úkol, který se začal zpracovávat už nebylo možné zastavit, lze například do ovládání operací zapojit uživatele.
\end{description}

\subsubsection{NSOperationQueue}

Obdobně jako GCD, je i zde potřeba určit kde a kdy se bude práce vykonávat.
Pro tento účel slouží fronta NSOperationQueue.
Tato fronta je implementována jako prioritní.
Tedy v momentě kdy je vložena operace s vyšší prioritou, bude vykonána dříve než stávající operace s nízkou prioritou (pokud to její závislosti dovolí).
Z tohoto odůvodu není zaručeno kdy se jednotlivé operace spustí.

Základní vytvoření operace a vložení do fronty je vidět na ukázce \ref{code:operation-queue-demonstration}.

\swiftcode{code:operation-queue-demonstration}{NSOperationQueue: Vytvoření fronty s operacemi}{assets/code/operation-queue-demonstration.swift}

\subsubsection{Prioritizace úkolů}

Z důvodu potřeby prioritizace některých úloh je možné jedntlivým operacím nastavit prioritu.
V takovém případě se vždy jako první vykonají operace s vyšší prioritou, následně se přistupuje k těm s nižším.

\subsubsection{Quality of Service}

Jednou z nejsilněších stránek NSOperationQueue je abstrakce front nad vlákny (tedy celého GCD).
V nastavení fronty lze zvolit stejně jako u GCD výchozí QoS.
Všechny operace pak budou vyžívat právě tuto prioritu.
Oproti GCD má však NSOperationQueue výhodu, že nepracuje pouze s jednou frontou.

To dává vývojáři možnost definovat QoS pro každou operaci zvlášť a to bez nutnosti vytvořát více front.
V závisloti na této vlastnosti je dále možné definovat, kolik operací může být \textit{maximálně} spuštěno v jeden okamžik.


\chapter{Testování}\label{testovani}

V kapitole o testování se podrobněji zabývám způsoby a postupy naznačenými v kapitolách \ref{analyza-mvvm} a \ref{vlakna-rac} o architektuře \acrshort{mvvm} a reaktivní programování.

Testováním software se rozumí postupy a procesy, pomocí kterých lze měřit, zda testovaný software (či jeho části) splňuje požadované nároky či nikoliv.
Opakovaným aplikováním těchto postupů lze v softwaru nalézt chyby, nedostatky nebo chybějící vlastnosti oproti dodané specifikaci.
Výsledky testování následně vypovídají o kvalitě softwaru a o míře splnění specifikace. \cite{software-testing-definition}

\bigskip

Jako techniku testování jsem zvolil testy chování (z anglického \texttt{behavior tests}).
Ty ověřují, zda aplikace na sadu vstupů vrací odpovídající výstupy.

\section{Testy chování}\label{testovani-bdd}

Chování modolu aplikace lze otestovat tak, že se zkoumá plnění jeho závazků.
Testované objekty mají definované rozhraní a závisloti, tím mají také deklarovány i závazky.
Závazky určují, jakým způsobem by měl objekt působit na zbylé části aplikace a jaké schopnosti a funkce má.
Ze schopností a funkcí lze odvodit jakým způsobem se má objekt chovat.

\subsection{Testování komponent}

Pomocí těchto testů lze ověřovat funkčnost libovolné komponenty aplikace.
Složitost a rozsah těchto testů se následně odvíjí od počtu závazků komponenty.
Testování probíhá tak, že vlastnosti objektu jsou pomocí rozhraní měněny.
Přitom se sleduje, jestli se objekt na základě změn chová podle očekávání. \cite{objcio-bdd}

\subsection{DSL}

Testy jsou popisovány takzvaným \acrshort{dsl} jazykem.
To je jazyk který pomocí kombinace klíčových slov a textového popisu definuje jak se komponenta má v určitou chvíli chovat.
Více o \acrshort{dsl} lze zjistit na \cite{petrikainulainen-dsl}.

V běžném testovacím rozhraní prostředí Xcode tento jazyk nenabízí.
Rozhodl jsem pro to využí dvou knihoven.
Pomocí Knihovny \textit{Quick} jsem mohl \acrshort{dsl} v projektu využít.
Očekávané hodnoty jsem ověřoval pomocí \textit{Nimble}.

\subsection{Testování ViewModelu}

Tento princip využívám ve své implementaci pro testování Modelu a ViewModelu.
Díky dostatečnému rozsahu testů logiky aplikace lze usuzovat, že v produkčním nasazení se vyskytne jen nepatrné množství chyb.
View vrstvu následně není samostatně nutné testovat, protože z testů ViewModelu je téměř jisté, že data jsou správně připravena k zobrazení.

U ViewModelu jsem podle zvoleného scénáře navrhl \acrshort{dsl}.
Tímto jazykem jsem definoval jak očekávám, že s objektem bude pracovat.
Pomocí Nimble knihovny jsem dále ověřoval, jestli na vytvořené vstupy vytváří ViewModel očekávané výstupy jak je vidět v ukázce \ref{code:bdd-dsl}.

Během implementace jsem vytvořil bez mála sto testů, díky kterým jsem odchytil velké množství chyb.
Tyto chyby jsem často do aplikace zanesl v momentě, kdy jsem k již hotové obrazovce implementoval novou funkcionalitu.

\swiftcode{code:bdd-dsl}{Použití \acrshort{dsl} pro definici testu}{assets/code/bdd-dsl.swift}

\subsection{Vývoj řízený testováním chování}

S pojmem testování chování je velmi blízce spjat \textit{vývoj řízený testováním chování} (z anglického Behavior Driven Development, zkráceně \acrshort{bdd}).
V tomto přístupu k vývoji se před konkrétní implementací nejdříve nadefinuje, jakým způsobem se mají testované komponenty chovat.
Následně se sestavují testy, které vyžadují implementování vyžadovaného chování.
Tím má objekt definované, jaké rozhraní musí implementovat a jaké závislosti bude vyžadovat.

Průběh vývoje probíhá cyklicky.
Nejdříve se nadefinují testy některé funkcionality, následně se implementuje jen tolik kódu, kolik je nutné aby testy byly úspěšné.
Tento postup se opakuje dokud nejsou implementovány veškeré funkce vyžadované po objektu.
Více informací o tomto přístupu je dostupných na \cite{objcio-bdd}.

Tento přístup jsem využil na část ViewModelů.
Vzhledem k rozsahu implementace jsem ale většinu testů napsal až po kompletní implementaci ViewModelu.


\section{Průběžná integrace}

Průběžná integrace (z anglického \textit{Continous integration}, zkráceně \texttt{CI}) je praktika vývoje software, při které členové týmu integrují svou práci mnohdy až několikrát denně.
Každá integrace je automaticky ověřena kompilací na buildovacím serveru a spuštěním testů.
Tato technika si klade za cíl odhalid integrační chyby aplikace co nejdříve je to možné.
To má za následek snížení časové náročnosti implemetace nových funkcí bez rizika narušení funkcionalit původní aplikace.

Nové funckionality se běžně skládají z nového kódu, upraveného produkčního kódu a upravených testů.
V momentě kdy vývojář označí funkcionalitu za kompletní, odešle své změny na vzdálený buildovací server, kde se aplikace zkompiluje a otestuje.
Integrace je následně provedena jen v případě, že celý proces proběhl bez chyby.
Jak je zjevné, při této technice je zásadní vysoké pokrytí aplikace testy.

Průběžné integrování je často velmi blízce vázáno na správu zdrojových kódů.
Některé systémy jako GitHub či GitLab je dokonce možné nastavit tak, aby by změny byly přidány do hlavní větve až ve chvíli, kdy build projde bez komplikací. \cite{travis-ci-building-pr}

Pro průběžnou integraci je na trhu dostupných mnoho řešení.
Rozhodoval jsem se mezi použitím komerčních řešení \textit{Travis CI} a \textit{Circle CI}.
První zmíněná možnost je v open-source komunitě velmi populární a dalo by se říct, že je pro \texttt{CI} téměř synonymem.
Podle nezávislého průzkumu z roku 2016 využívá \textit{Travis CI} téměř pětinásobek uživatelů než ostatních poskytovatelů dohromady. \cite{oregonstate-ci-survey}

Druhé řešení jsem chtěl vyzkoušet kvůli rostoucí popularitě. \cite{circleci-popularity}
Bohužel je kompilace na operačním systému macOS pro open-source projekty možná až po individuálním schválení. \cite{circleci-pricing}
Z tohoto důvodu jsem nakonec zvolil první řešení, kde je kompilace na systému macOS pro studenty zdarma.

\subsection{Statická analýza kódu}

Spolu s roustoucím týmem a rostoucím zdrojovým kódem se často zavádí směrnice programování.
Tyto směrnice udávají jakým způsobem je kód formátovaný.
Při dodržování těchto směrnic se kód stává konzistentním a dobře čitelným, to má za následek zrychlení vývoje a zvyšuje udržitelnost kódu.
Aby nebylo nutné analyzovat kód ručně, existují nástroje které analýzu automatizují.

Pro zajištění konzistence jsem použil nástroj \texttt{SwiftLint} od společnosti \texttt{Provider}.
\texttt{SwiftLint} je nástroj obsahující sadu pravidel pro statickou analýzu kódu.
Mimo jiné kontroluje správné zarovnání kódu, délky řádků či názvy proměnných.
V době psaní práce obsahuje tento nástroj dohromady téměř osmdesát pravidel.
Ve své implemetaci jsem se rozhodl vynechat dvě pravidla, které jsem nepovažoval za stěžejní a jejich dodržování by z mého pohledu vedlo ke snížení čitelnosti.
Kromě těchto dvou pravidel jsem využil i možnost vypnout pravidla pro určité části kódu.
Nejčastěji se jednalo o pravidla na kontrolu délky metod.

Tuto analýzu zajišťuje buildovací server před začátkem kompilace.
Pokud nejsou v kódu závažné porušení směrnice, pokračuje se ke kompilaci.
V opačném případě nastane chyba a je nutné kód opravit a integraci pustit znovu.


\section{Průběžné doručování}

Průběžné doručování (z anglického \textit{Continuous delivery}, zkráceně \textit{CD}) je způsob, kterým lze nasadit nové funkcionality, změny konfigurace či opravy chyb do produkčního prostředí.
Cílem této praktiky je automatizovaně aplikovat opakované postupy potřebné k vydání aplikace.
Tím lze minimalizovat množství chyb, které by vývojář jinak mohl při vydávání způsobit.
Vývojáři stačí v tomto případě pouze doručování spusti, server zařídí aby aplikace byla správně nasazena.

Kromě minimalize výskytu chyb tato technika snižuje čas vývoje.
Díky automatizovanému vydávání mají vývojáři více času na samotný vývoj.
Tím lze zaručit vyšší kvalitu kódu, nižší náklady na vývoj a také častější aktualizace software. \cite{continuousdelivery-what-is-ci}

Přestože pro svou práci nemám reálné produkční prostředí, rozhodl jsem se \textit{CD} použít pro distribuci testovací verze aplikace.
Díky tomu jsem mohl jednotlivé verze vydávat v průběhu vývoje velmi rychle a ověřit funkčnost aplikace na skutečných zařízeních.
Vydání verze jsem nastavil na každé nahrání zdrojových kódů do repozitáře.
Po statické analýze a kompilaci byla aplikace nahrána do prostředí Fabric, odkud si ji mohli registrovaní testeři stáhnout.

\section{Lokální testování}

Protože aplikace komunikuje výhradně s tiskárnou, musí být tiskárna dostupná i během vývoje.
Abych nemusel tiskárnu pořizovat, využil jsem možnost vytvořit virtuální tiskárnu.

S využitím technologie Docker jsem vytvořil kontejner s nainstalovanou aplikací OctoPrint.
Pomocí konfiguračního souboru jsem povolil vytvoření virtuální tiskárny.

OctoPrint jsem takto mohl mít spuštěný lokálně na počítači.
Po připojení OctoPrint k virtuálnímu portu tiskárny jsem mohl vyvíjet aplikaci i bez nutnosti vlastnit tiskárnu.

V ukázce \ref{code:docker} demonstruji spuštění kontejneru s virtuální tiskárnou z příkazové řádky svého počítače.

\begin{listing}[htbp]
\caption{\label{code:docker}Spuštění virtuální tiskárny pomocí Docker}
\begin{minted}[linenos, bgcolor=bgcode]{bash}
docker run -p"3200:5000" josefdolezal/virtuprint-docker 
\end{minted}
\end{listing}


\begin{conclusion}
  Cílem práce bylo vypracování mobilní aplikace pro platformu iOS umožňující nastavovat 3D tiskárny a tisknout z nich.

Výsledná aplikace umožňuje zobrazit přehled aktuálního tisku, správu souborů a nastavení 3D tiskárny.
Pomocí aplikace je také možné sledovat video stream z webové kamery připojené k tiskárně.
Uživatel také může vybrat soubor ve svém zařízení, nahrát ho do tiskárny a následně vytisknout.

V aplikaci lze komunikovat s libovolnou tiskárnou dostupnou na místní síti, která je aplikací automaticky nalezena.
Uživatel může velmi jednoduše takto tiskárnu přidat a obsloužit celý tisk.

\bigskip

Během implementace jsem se snažil většinu pozornosti věnovat architektuře aplikace.
Mnoho částí jsem proto v průběhu implementace několikrát přepsal.

Velkou výzvou bylo implementovat aplikaci reaktivně.
S reaktivním programování jsem doposud neměl žádné zkušenosti.
Nyní jsem ale rád, že jsem se takto rozhodl.
Využití reaktivního přístupu mi ušetřilo mnoho práce a velmi pravděpodobně i mnoho chyb, které bych jinak standardní cestou do aplikace zanesl.

\bigskip

Jsem přesvědčený, že díky využití zvolené architektury jsem také zvýšil testovatelnost aplikace.
Přesto, že jsem testy aplikace nikdy dříve nepsal, nebylo složité je s toutou architekturou vytvořit.

\subsection*{Výhled do budoucna}

V budoucnu je možné do vývoje aplikace přizvat programátory pohybující se pravidelně v prostředí 3D tisku a rozšířit ji tak v komunitě.
Z tohoto důvodu jsou zdrojové kódy práce volně dostupné jako open source.
Aplikaci je díky tomu možné rozšířit o nové funkcionality v případě aktualizace \acrshort{api} OctoPrint.

V současné době ale neplánuji aplikaci distrubuovat běžným uživatelům obchodem App Store.

\end{conclusion}

\bibliographystyle{csn690}
\bibliography{bibliography}

\appendix

\chapter{Seznam použitých zkratek}
% \printglossaries
% \begin{description}
%
% \end{description}

\chapter{Obsah přiloženého CD}

%upravte podle skutecnosti

\begin{figure}
	\dirtree{%
		.1 readme.txt\DTcomment{stručný popis obsahu CD}.
		.1 exe\DTcomment{adresář se spustitelnou formou implementace}.
		.1 src.
		.2 impl\DTcomment{zdrojové kódy implementace}.
		.2 thesis\DTcomment{zdrojová forma práce ve formátu \LaTeX{}}.
		.1 text\DTcomment{text práce}.
		.2 thesis.pdf\DTcomment{text práce ve formátu PDF}.
		.2 thesis.ps\DTcomment{text práce ve formátu PS}.
	}
\end{figure}

\end{document}
