3D tisk je proces, při kterém lze z digitálního modelu vytvořit reálný objekt.
Tyto objekty vznikají postupným nanášením jednotlivých vrstev modelu na sebe ve vertikálním směru.
Jedna z metod využívajících vrstvení materiálu se nazývá Fused deposition modeling (zkráceně FDM).
Ve své bakalářské práci uvažuji právě tuto metodu.
Princip FDM spočívá v tavení plastu (případně jiných materiálů) pomocí tiskové hlavy, která následně nanáší taveninu ve vrstvách na sebe.
Takto vytvořené plastové objekty mají velkou výhodu v nízkých nákladech na výrobu.
Jsou tedy ideálním prostředkem k výrobě prototypů nebo produkování omezeného množství výrobků.
Tisk lze obsloužit pomocí mnoha aplikačních rozhraní.
Velmi oblíbeným nástrojem je OctoPrint, který nabízí ovládání pomocí webového prohlížeče.
Toto řešení ale umožňuje správu jen jedné tiskárny, navíc je přizpůsobené pouze pro počítače.
Přestože trend tisknutí reálných předmětů stále roste, neexistuje v současné době jednoduchá a volně dostupná mobilní aplikace pro systém iOS, která by umožnila ovládání tiskárny bez použití počítače.
S ohledem na stoupající oblíbenost mobilních aplikací přestává být současné webové rozhraní dostatečně pohodlné.
Uživatelé chtějí mít rychlý přehled o stavu svých tiskáren z jednoho místa a bez nutnosti prokazování se heslem.
Projekt má za cíl vytvořit mobilní aplikaci, která uživatelům usnadní práci s tiskárnou.
Vzhled a ovládání aplikace budou zvlášť přizpůsobené chytrým telefonům, ale i tabletům.
Podporován bude operační systémem iOS.
Mezi hlavní cíle patří nejen umožnit komunitě s aplikací pracovat, ale také jí dát možnost kód libovolně upravit.
Z tohoto důvodu je práce vedena jako open source.
